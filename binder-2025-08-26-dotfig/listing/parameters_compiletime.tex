\section{Compile Time Parameters}

% create command to draw operation tables
\newcommand{\paramcompiletimedef}[2]{
    ~\\ \begin{tabularx}{\linewidth}{|
        >{\columncolor[HTML]{C0C0C0}}l |X|}
        \hline
        Type & \texttt{#1} \\ \hline
        Value   & \texttt{#2} \\ \hline
    \end{tabularx}
}

This section enumerates simulation parameters and provides default settings that were used.

See \url{https://github.com/mmore500/dishtiny/blob/prq49/include/dish2/spec/Spec_prq49.hpp} for source code defining compile time parameters.

\subsection{NLEV}

\paramcompiletimedef{size\_t}{2}

How many hierarchical kin group levels should be simulated?

\subsection{AMT\_NOP\_MEMORY}

\paramcompiletimedef{size\_t}{4}

How many nop and transient nop states should exist in the peripheral?

\subsection{STATE\_EXCHANGE\_CHAIN\_LENGTH}

\paramcompiletimedef{size\_t}{128}

How many callees should we displace state by in state exchange experiments?

\subsection{sgpl\_spec\_t::num\_cores}

\paramcompiletimedef{size\_t}{32}

How many virtual cores should each cardinal processor's virtual CPU be able to support?

\subsection{sgpl\_spec\_t::num\_fork\_requests}

\paramcompiletimedef{size\_t}{3}

How many fork requests can a virtual core make at most?

\subsection{sgpl\_spec\_t::num\_registers}

\paramcompiletimedef{size\_t}{8}

How many registers should each virtual core contain?

\subsection{sgpl\_spec\_t::switch\_steps}

\paramcompiletimedef{size\_t}{8}

Maximum num steps executed on one core before next core is executed.

\subsection{sgpl\_spec\_t::global\_matching\_t}

\paramcompiletimedef{typedef}{%
emp::MatchDepository<\newline
\hspace*{2em}// \mbox{program index type} \newline
\hspace*{2em}unsigned short, \newline
\hspace*{2em}// \mbox{match metric} \newline
\hspace*{2em}emp::OptimizedApproxDualStreakMetric<64>,\newline
\hspace*{2em}// \mbox{match selector} \newline\hspace*{2em}emp::statics::RankedSelector<\newline
\hspace*{4em}// \mbox{match threshold} \newline
\hspace*{4em}std::ratio<1, 5>\newline
\hspace*{2em}>,\newline
\hspace*{2em}// \mbox{regulator} \newline
\hspace*{2em}emp::PlusCountdownRegulator<\newline
\hspace*{4em}std::deci, // Slope\newline
\hspace*{4em}std::ratio<1,4>, // MaxUpreg\newline
\hspace*{4em}std::deci, // ClampLeeway\newline
\hspace*{4em}2 // CountdownStart\newline
\hspace*{2em}>,\newline
\hspace*{2em}true, // raw caching \newline
\hspace*{2em}8 // regulated caching\newline
>
}

What matching datastructure implementation should we use for global jump tables?

\subsection{sgpl\_spec\_t::local\_matching\_t}

\paramcompiletimedef{typedef}{%
emp::MatchDepository<\newline
\hspace*{2em}// \mbox{program index type} \newline
\hspace*{2em}unsigned short, \newline
\hspace*{2em}// \mbox{match metric} \newline
\hspace*{2em}emp::OptimizedApproxDualStreakMetric<64>,\newline
\hspace*{2em}// \mbox{match selector} \newline\hspace*{2em}emp::statics::RankedSelector<\newline
\hspace*{4em}// \mbox{match threshold} \newline
\hspace*{4em}std::ratio<1, 2>\newline
\hspace*{2em}>,\newline
\hspace*{2em}// \mbox{regulator} \newline
\hspace*{2em}emp::PlusCountdownRegulator<\newline
\hspace*{4em}std::deci, // Slope\newline
\hspace*{4em}std::ratio<1,4>, // MaxUpreg\newline
\hspace*{4em}std::deci, // ClampLeeway\newline
\hspace*{4em}2 // CountdownStart\newline
\hspace*{2em}>,\newline
\hspace*{2em}false, // raw caching \newline
\hspace*{2em}0 // regulated caching\newline
>
}

What matching datastructure implementation should we use for local jump tables?

\subsection{sgpl\_spec\_t::tag\_width}

\paramcompiletimedef{size\_t}{64}

Tag width in bits.
