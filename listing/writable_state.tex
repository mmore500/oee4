% create command to draw state tables
\newcommand{\writablestatedef}[2]{
    \begin{tabular}{|
        >{\columncolor[HTML]{C0C0C0}}l |l|}
        \hline
        Type & \texttt{#1} \\ \hline
    \end{tabular} \\
}

\section{Writable State}

Writable state refers to the collection of output values that evolving programs running within each cardinal processor can write to and read from.
Some of these outputs enable interaction with the simulation (i.e., control phenotypic characteristics).
Each cardinal processor has an independent copy of each piece of writable state state.

See \url{https://github.com/mmore500/dishtiny/tree/prq49/include/dish2/peripheral/readable_state/writable_state} for source code implementing writable state.

\subsection{Nop State ($4\times$)}

\writablestatedef{float}

Writing to this state has no external effect.
It can be used as global memory shared between cores.

\subsection{Transient Nop State ($4\times$)}

\writablestatedef{float}

Writing to this state has no external effect.
It is cleared regularly by the decay to baseline service.
It can be used as temporary global memory shared between cores.

\subsection{Apoptosis Request}

\writablestatedef{char}

Writing a nonzero value to this state causes cell death.

\subsection{Heir Request}

\writablestatedef{char}

If this state is set when cell death occurs, the cardinal processor's neighbor cell will inherit leftover resource from the cell's stockpile.

\subsection{RepLev Request (0 thru $L$)}

\writablestatedef{char}

Controls kin group inheritance for daughter cells spawned to this cardinal processor's neighbor tile.

If no copies of this state are set at cell spawn, the daughter cell will have no common kin group IDs.
If one copy of this state is set at cell spawn, the daughter cell will have one common kin group ID.
If $L$ copies of this state are set at cell spawn, the daughter cell will have $L$ common kin group IDs.

\subsection{Resource Receive Resistance}

\writablestatedef{float}

Setting this state reduces the amount of resource received from the cardinal processor's neighbor cell.

\subsection{Resource Reserve Request}

\writablestatedef{float}

Setting this state prevents that amount of stockpiled resource from being drawn from to be sent to the cardinal processor's neighbor cell.

\subsection{Resource Send Limit}

\writablestatedef{float}

Setting this state caps the amount of resource that this cell can send to the cardinal processor's neighbor cell per update.

\subsection{Resource Send Request}

\writablestatedef{float}

Setting this state initiates resource sharing to the cardinal processor's neighbor cell.
The value stored controls the amount of resource shared.

\subsection{Spawn Arrest}

\writablestatedef{char}

Setting this state prevents this cell from spawning offspring into this cardinal processor's neighbor tile, even if sufficient resource is available.

\subsection{Spawn Request}

\writablestatedef{char}

Setting this state attempts to initiate spawning offspring into this cardinal processor's neighbor tile.
