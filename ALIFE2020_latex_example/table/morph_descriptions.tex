\newcommand{\includesnapshot}[1] {%
\adjustbox{trim={0.05\width} {0.35\width} {0.05\width} {0.35\width},clip}%
    {\includegraphics[height=0.3\textheight]{#1}}
}
\newcommand{\tinyurl}[1]{\tiny\url{#1}}
\newcommand{\morphtext}[1] {%
\color[HTML]{FFFFFF} \huge \raisebox{1.2em}{\textbf{#1}}%
}
\newcommand{\videolink}[1] {%
\raisebox{2.8em}{\begin{minipage}{1.3 cm} \tinyurl{#1} \end{minipage}}
}

{
\catcode`\\%=12
\begin{table}[]
\begin{tabular}{clll}
\multicolumn{1}{l}{\textbf{Morph}}               & \textbf{Description} & \textbf{Snapshot} & \textbf{Snapshot Link} \\
\cellcolor[HTML]{4C72B0}{\color[HTML]{FFFFFF} a} &  \includegraphics[width=0.2\linewidth]{\detokenize{snapshots/a=kin-group-id+idx=0+proc=0+series=16005+stint=0+thread=0+update=28991+_endeavor=16+_repro=rvV3h5Ru0tlNBvox+_slurm_job_id=24414678+_source=8eb7a4e-dirty+_treatment=bucket\%prq49~diversity\%0.50_series~mut_freq\%1.00~mut_sever\%1.00+ext=}}             & \url{https://youtu.be/-SvgSmIsPQc?t=0} & This morphology consists of indidivual cells. There are no multicellular orgnamisms visible. Resource use is low---most cells simply hoard resource until their stockpile is greater than one and then stop collecting it. In fact, only a handful of cells keep spending resource continusously.              \\
\cellcolor[HTML]{DD8452}{\color[HTML]{FFFFFF} b} & \includegraphics[width=0.2\linewidth]{\detokenize{snapshots/a=kin-group-id+idx=0+proc=0+series=16005+stint=1+thread=0+update=14271+_endeavor=16+_repro=hVlcCQPvlFIR0ckX+_slurm_job_id=25050103+_source=819521e-dirty+_treatment=bucket\%prq49~diversity\%0.50_series~mut_freq\%1.00~mut_sever\%1.00+ext=}}                 & \url{https://youtu.be/-SvgSmIsPQc?t=60} & This morphology consists of mostly individual cells, with some two-, three-, and four-cell groups evenly spread out. Resource is used very consistently. Most groups have a lower amount of resource in their stockpiles than individual cells, but it is still relatively high.               \\
\cellcolor[HTML]{55A868}{\color[HTML]{FFFFFF} c} & \includegraphics[width=0.2\linewidth]{\detokenize{snapshots/a=kin-group-id+idx=0+proc=0+series=16005+stint=2+thread=0+update=17471+_endeavor=16+_repro=fHVOHaP3ZgZwIfuP+_slurm_job_id=25050099+_source=819521e-dirty+_treatment=bucket\%prq49~diversity\%0.50_series~mut_freq\%1.00~mut_sever\%1.00+ext=}}                  & \url{https://youtu.be/-SvgSmIsPQc?t=120} & Clear multicellular groups are visible. Each group consists of at least 50 cells, with the average being around 300. As such, cells are clumped together into giant blob-like groups. Most of these groups try to maintain a lot of resource in their stockpile. It seems that every group is getting punished by the simulation by having some of its cells mutate to belong into different groups, but this is not really affecting them due to their size.              \\
\cellcolor[HTML]{C44E52}{\color[HTML]{FFFFFF} d} & \includegraphics[width=0.2\linewidth]{\detokenize{snapshots/a=kin-group-id+idx=0+proc=0+series=16005+stint=14+thread=0+update=16959+_endeavor=16+_repro=BnqHEceSUrehEdlE+_slurm_job_id=25086959+_source=819521e-dirty+_treatment=bucket\%prq49~diversity\%0.50_series~mut_freq\%1.00~mut_sever\%1.00+ext=}}                  & \url{https://youtu.be/-SvgSmIsPQc?t=840} & Clear groups are visible. These are relatively small --- around 10 to 15 cells in size. Some group nesting is present in certain groups, but these enclaved groups do not seem to be competing with the surrounding groups for space. Resource use is low among bigger groups but higher for the nested groups.              \\
\cellcolor[HTML]{8172B3}{\color[HTML]{FFFFFF} e} & \includegraphics[width=0.2\linewidth]{\detokenize{snapshots/a=kin-group-id+idx=0+proc=0+series=16005+stint=15+thread=0+update=16639+_endeavor=16+_repro=HJOcggiTrmZJVbXg+_slurm_job_id=25086417+_source=819521e-dirty+_treatment=bucket\%prq49~diversity\%0.50_series~mut_freq\%1.00~mut_sever\%1.00+ext=}}                  & \url{https://youtu.be/-SvgSmIsPQc?t=905} & Groups are visibly elongated along the horizontal axis. Resource use is relatively stable, with most groups maintaining a stockpile smaller than one. After some time, groups begin to spread vertically while maintaining their horizontality.             \\
\cellcolor[HTML]{937860}{\color[HTML]{FFFFFF} f} & \includegraphics[width=0.2\linewidth]{\detokenize{snapshots/a=kin-group-id+idx=0+proc=0+series=16005+stint=39+thread=0+update=10303+_endeavor=16+_repro=kk8MYN0ff7wGODOC+_slurm_job_id=25091828+_source=819521e-dirty+_treatment=bucket\%prq49~diversity\%0.50_series~mut_freq\%1.00~mut_sever\%1.00+ext=}}                  & \url{https://youtu.be/-SvgSmIsPQc?t=2004} & Groups form horizontal groups similar to morphology (e). However, their horizontal stripes are "offset". Resource use is less than one in the center of the groups, and greater than one on their horizontal edges.               \\
\cellcolor[HTML]{DA8BC3}{\color[HTML]{FFFFFF} g} & \includegraphics[width=0.2\linewidth]{\detokenize{snapshots/a=kin-group-id+idx=0+proc=0+series=16005+stint=45+thread=0+update=12991+_endeavor=16+_repro=0W10szaxUWV60t5Q+_slurm_job_id=25092156+_source=819521e-dirty+_treatment=bucket\%prq49~diversity\%0.50_series~mut_freq\%1.00~mut_sever\%1.00+ext=}}                  & \url{https://youtu.be/-SvgSmIsPQc?t=2275} & Initially, clear similarities to (e) can be seen. Groups are clearly horizontal, taking up only one row of cells. However, their morphology seems to spontaneously transform based on a timer to be more blob-like; the horizontal streaks are offset vertically every other cell, and form a sort of "zig-zag" pattern. Resource use is low around the edges but high near the center of the groups.                   \\
\cellcolor[HTML]{8C8C8C}{\color[HTML]{FFFFFF} h} & \includegraphics[width=0.2\linewidth]{\detokenize{snapshots/a=kin-group-id+idx=0+proc=0+series=16005+stint=59+thread=0+update=11839+_endeavor=16+_repro=PrKgu7JjxV9bJooq+_slurm_job_id=25092244+_source=819521e-dirty+_treatment=bucket\%prq49~diversity\%0.50_series~mut_freq\%1.00~mut_sever\%1.00+ext=}}                  & \url{https://youtu.be/-SvgSmIsPQc?t=2948} & Initially, groups are extremely horizontal. They consist almost fully of a single row, maintaining a striking similarity to (e). However, they spontaneously "explode" into blobs that compete with others for space. As such, their edges are highly motile and their resource use is extremely high.                   \\
\cellcolor[HTML]{CCB974}{\color[HTML]{FFFFFF} i} & \includegraphics[width=0.2\linewidth]{\detokenize{snapshots/a=kin-group-id+idx=0+proc=0+series=16005+stint=74+thread=0+update=12991+_endeavor=16+_repro=r5bt4vkTWWZvyaKe+_slurm_job_id=25092628+_source=819521e-dirty+_treatment=bucket\%prq49~diversity\%0.50_series~mut_freq\%1.00~mut_sever\%1.00+ext=}}                  & \url{https://youtu.be/-SvgSmIsPQc?t=3740} & Groups form tetris-like pieces: they mostly consist of less than ten cells arranged in a geometric manner. Some motility is seen, with groups prefering to spawn in a single direction --- they have a "head" and a "tail". Resource use is incredibly high, with most groups only being able to maintain a stockpile greater than one for a few updates.              \\
\cellcolor[HTML]{64B5CD}{\color[HTML]{FFFFFF} j} & \includegraphics[width=0.2\linewidth]{\detokenize{snapshots/a=kin-group-id+idx=0+proc=0+series=16005+stint=100+thread=0+update=8767+_endeavor=16+_repro=wSdyWubKcpinRb7H+_slurm_job_id=24415753+_source=8eb7a4e-dirty+_treatment=bucket\%prq49~diversity\%0.50_series~mut_freq\%1.00~mut_sever\%1.00+ext=}}                  & \url{https://youtu.be/-SvgSmIsPQc?t=4980} & Initially, groups are incredibly similar to (e): highly horizontal, and only using resource at their extremities. However, half of them spontaneously transform into blob-like groups. Resource use is low across the whole morphology.
\end{tabular}

\caption{
Qualitative morph phenotype categorizations.
Color coding of morph IDs has no significance beyond guiding the eye in scatter plots where points are labeled by morph.
Snapshot visualizes spatial layout of kin groups on toroidal grid at a fixed point in time.
Each cell corresponds to a small square tile.
Color hue denotes and black borders divide outermost kin groups while color saturation denotes and white borders divide innermost kin groups.
}
\label{tab:morph_descriptions}

\end{table*}
}