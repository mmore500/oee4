\section{Conclusion}

This case study shines anecdotal light on a loose coupling between novelty, complexity, and adaptation.

We observe instances where novelty coincides with adaptation and instances where it does not. We observe instances where increases in complexity coincide with adaptation and where decreases in complexity coincide with adaptation. We observe instances where innovation coincides with spikes in complexity and instances where it does not. We even observe contradiction between metrics that measure different aspects functional complexity, with a near tripling of interface complexity coinciding with a drop in critical fitness complexity.

The anecdotal evidence of loose coupling between the conceptual threads of novelty, complexity, and adaptation provided by this case study highlights the importance of parsing out these factors independently when studying open-ended evolution --- their  correlation is by no means for granted.
Future work exploring open-endedness should continue along the lines of the MODES framework \citep{dolson2019modes}, building a broad analytical toolbox that can distinguish disparate dimensions of open-endedness.

% Continuing the work of MODES to broaden a toolbox rather than focus in on a single metric.

% Serves as a trial balloon for future, more systematic exploration of the relationship between novelty, complexity, and adaptation.

% For example, if that only activated in response to the other lineage then couldn't detect that

% Especially in systems where fitness is implicit it's costly or impractical to do comprehensive analyses.
% We need to continue develop methodology to statisticaly sample 

% Problem also that fitness depends on conditions, so there may be problems replicating conditions of evolution environment (presence of mutation, presence of other strains).
% Real-time proof-of-concept but requires more thought about methodology to control for the slowing of evolutionary pace.


% We use the idea of in addition to existing ideas about sequence complexity.
% Interface complexity the richness of interactions between an agent and its environment.
% 8 After all, the central concept of adaptation—``the design or suitability of an object for a particular
% function'' (Gould 2002, p. 117)—directly refers to functionality. A few examples of those taking
% biological functionality as integral to biological complexity include Bronowski (1970), Wicken (1979),
% Dawkins (1986), Kampis and Csa´nyi (1987), Carroll (2001), Adami (2002) \citep{korb2011evolution}

% The question of the relation between sequence complexity and functional complexity is.
% Sequence complexity is defined in terms of entropy of , but in a practical sense it can be approximated by looking at critical fitness complexity.

% a novel innovation is accompanied by a decrease in sequence/critical fitness complexity but an increase in interface complexity.

% Already known in Avida, where sequence/critical fitness complexity would decrease after a innovation as became more adapted 

% Surprisingly, the novelty wasn't accompanied by out-of-the ordinary fitness differential.

% However, we present a case study where 

% If $d$ is a member of another lineage, then composite fitness complexity didn't increase greatly even though interface complexity more than doubled.

% A second novelty occurs without a clear signal with respect to sequence/fitness complexity or interface complexity.

% Methodology to quantify is particularly critical.
% One area of interest is understanding the functional complexity of .
% Phenotype complexity, Interface complexity and statistical methodololgy to sample stuff.
% We advance proposed methodology by dolson et al with respect to identifying

% We look at a case study to see how these complexity metrics change over an evolutionary history with qualitative moprhological innovations. 

