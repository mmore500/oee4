\section{Discussion}

Throughout the case study lineage, we describe ten qualitatively distinct multicellular morphologies (Figure \ref{tab:morph_descriptions}).
The emergence of some, but not all, of these morphologies coincided with an increase in fitness compared to the preceding population.
For example, morphologies $c$ and $f$ do not significantly outcompete their predecessors while morphologies $d$, $e$, and $g$ do (Figure \ref{fig:fitness:fitness_neutrality}).
This latter set of novelties might be described as ``innovations,'' which Hochberg et al. define as qualitative novelty associated with an increase in fitness \citep{hochberg2017innovation}.
Interestingly, the magnitude of the fitness differentials associated with the emergence of morphologies $d$, $e$, and $g$ do not appear to fall outside the bounds of other stint-to-stint fitness differentials (Figure \ref{fig:fitness:fitness_magnitude}).

The relationship between innovation and complexity also appears to be loosely coupled.
The emergence of morphology $d$ was accompanied by a spike in critical fitness complexity (from 25 sites at stint 13 to 43 sites at stint 14).
However, the emergence of morphology $e$ may have coincided with a loss of critical fitness complexity (from 43 sites to 31 sites).
Due to limitations in our phylogenetic tracking, is unclear whether morphology $d$ was a stepping stone to the emergence of morphology $e$.
If it was not, the emergence of morphology $e$ coincided with a more modest increase in fitness complexity from 25 sites to 31 sites.
Similarly, the emergence of morphology $g$ with 42 critical sites at stint 45 coincided with a relatively modest increase in fitness complexity from 39 critical sites at stint 44.

We also see evidence that increases in complexity do not imply qualitative novelty.
In Figure \ref{fig:fitness_complexity:critical_fitness_complexity}, we can also observe notable increases in critical fitness complexity that did not coincide with apparent morphological innovation.
For example, fitness complexity spiked from 11 sites at stint 11 to 27 sites at stint 12 while morphology $b$ was retained.
In addition, a more gradual increase in fitness complexity was observed from 27 sites at stint 16 to 46 sites at stint 36 all under the guise of morphology $e$.

Finally, we also observed surprising contradictions between alternate measures of functional complexity.
Notably, cardinal interface complexity more than doubled from 6 interactions under morph $d$ to 17 interactions with the emergence of morph $e$.
However, criticial fitness complexity of morph $d$ was 12 sites greater than morph $e$ at stint 15.
In addition, the gradual increase in critical fitness complexity between stint 15 and 36 under moprhology $e$ is not accompanied by a clear change in interface complexity.
These apparent inconsistencies between metrics for functional complexity evidence the multidimensionality of this idea and underscore well-known difficulties in attempts to describe and quantify it \citep{bottcher2018molecules}. 