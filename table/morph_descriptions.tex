% \pragmaonce

% adapted from https://www.overleaf.com/learn/latex/Commands
\providecommand{\dissertationelse}[2]{%
% adapted from https://tex.stackexchange.com/a/33577
\ifdefined\DISSERTATION
#1
\else
#2
\fi
}

% \pragmaonce

% adapted from https://www.overleaf.com/learn/latex/Commands
\providecommand{\dissertationexclude}[1]{%
% adapted from https://tex.stackexchange.com/a/33577
\ifdefined\DISSERTATION
\else
#1
\fi
}

% \pragmaonce

% adapted from https://www.overleaf.com/learn/latex/Commands
\providecommand{\dissertationonly}[1]{%
% adapted from https://tex.stackexchange.com/a/33577
\ifdefined\DISSERTATION%
#1%
\else%
\fi
}


\newcommand{\includesnapshot}[1] {%
\adjustbox{trim={0.05\width} {0.35\width} {0.05\width} {0.35\width},clip}%
    {\includegraphics[height=\dissertationexclude{0.3}\dissertationonly{0.25}\textheight]{#1}}
}
\newcommand{\morphtext}[1] {%
\color[HTML]{FFFFFF} \huge \raisebox{1.2em}{\textbf{#1}}%
}
\newcommand{\videolink}[1] {%
\raisebox{
\dissertationexclude{2.8em}
\dissertationonly{2.2em}
}{\begin{minipage}{\dissertationelse{2.4cm}{1.3 cm}} \dissertationelse{\fontsize{6}{7}\selectfont}{\tiny}\url{#1} \end{minipage}}
}
\newcommand{\descript}[1] {%
\raisebox{
  \dissertationexclude{2.8em}
  \dissertationonly{2.2em}
}{\begin{minipage}{\linewidth}\dissertationonly{\fontsize{8}{8}\selectfont} #1 \end{minipage}}
}


{
\catcode`\%=12
\begin{table*}
\begin{tabular}{cp{0.4\textwidth}ll}
\multicolumn{1}{l}{\textbf{ID}}               & \textbf{Morphology} & \textbf{Snapshot} & \textbf{Video} \\
\cellcolor[HTML]{4C72B0}{ \morphtext{a} } & \descript{Individual cells, no multi-cellular kin groups. Resource use is low---most cells simply hoard resource until their stockpile is beyond sufficient to reproduce. Only a handfuls of cells intermittently expend resource.} & \includesnapshot{\detokenize{snapshots/a=kin-group-id+idx=0+proc=0+series=16005+stint=0+thread=0+update=28991+_endeavor=16+_repro=rvV3h5Ru0tlNBvox+_slurm_job_id=24414678+_source=8eb7a4e-dirty+_treatment=bucket%prq49~diversity%0.50_series~mut_freq%1.00~mut_sever%1.00+ext=.png}}             &  \videolink{https://hopth.ru/21/b=prq49+s=16005+t=0+v=video+w=specimen}          \\
\cellcolor[HTML]{DD8452}{\morphtext{b}} & \descript{Mostly individual cells, with some two-, three-, and four-cell groups evenly spread out. Resource usage occurs in short spurts in one or two adjacent cells. } & \includesnapshot{\detokenize{snapshots/a=kin-group-id+idx=0+proc=0+series=16005+stint=1+thread=0+update=14271+_endeavor=16+_repro=hVlcCQPvlFIR0ckX+_slurm_job_id=25050103+_source=819521e-dirty+_treatment=bucket%prq49~diversity%0.50_series~mut_freq%1.00~mut_sever%1.00+ext=.png}}                 & \videolink{https://hopth.ru/21/b=prq49+s=16005+t=1+v=video+w=specimen}              \\
\cellcolor[HTML]{55A868}{\morphtext{c}} & \descript{Large multi-cellular groups dominate, consisting of hundreds of cells. Group growth is unchecked and continues until cells' resource stockpiles are entirely depleted by the excess group size penalty.} & \includesnapshot{\detokenize{snapshots/a=kin-group-id+idx=0+proc=0+series=16005+stint=2+thread=0+update=17471+_endeavor=16+_repro=fHVOHaP3ZgZwIfuP+_slurm_job_id=25050099+_source=819521e-dirty+_treatment=bucket%prq49~diversity%0.50_series~mut_freq%1.00~mut_sever%1.00+ext=.png}}                  & \videolink{https://hopth.ru/21/b=prq49+s=16005+t=2+v=video+w=specimen}             \\
\cellcolor[HTML]{C44E52}{\morphtext{d}} & \descript{Clear groups of 10 to 15 cells in size form. Cell proliferation appears somewhat more active at the periphery of groups compared to the interior.} & \includesnapshot{\detokenize{snapshots/a=kin-group-id+idx=0+proc=0+series=16005+stint=14+thread=0+update=16959+_endeavor=16+_repro=BnqHEceSUrehEdlE+_slurm_job_id=25086959+_source=819521e-dirty+_treatment=bucket%prq49~diversity%0.50_series~mut_freq%1.00~mut_sever%1.00+ext=.png}}                  & \videolink{https://hopth.ru/21/b=prq49+s=16005+t=14+v=video+w=specimen}               \\
\cellcolor[HTML]{8172B3}{\morphtext{e}} & \descript{Groups are visibly elongated along the horizontal axis. After initial development, some gradual, irregular growth occurs along the vertical axis.} & \includesnapshot{\detokenize{snapshots/a=kin-group-id+idx=0+proc=0+series=16005+stint=15+thread=0+update=16639+_endeavor=16+_repro=HJOcggiTrmZJVbXg+_slurm_job_id=25086417+_source=819521e-dirty+_treatment=bucket%prq49~diversity%0.50_series~mut_freq%1.00~mut_sever%1.00+ext=.png}}                  & \videolink{https://hopth.ru/21/b=prq49+s=16005+t=15+v=video+w=specimen}              \\
\cellcolor[HTML]{937860}{\morphtext{f}} & \descript{Groups are horizontally elongated similarly to morphology $e$, but have a greater consistent vertical thickness of three or four cells.} & \includesnapshot{\detokenize{snapshots/a=kin-group-id+idx=0+proc=0+series=16005+stint=39+thread=0+update=10303+_endeavor=16+_repro=kk8MYN0ff7wGODOC+_slurm_job_id=25091828+_source=819521e-dirty+_treatment=bucket%prq49~diversity%0.50_series~mut_freq%1.00~mut_sever%1.00+ext=.png}}                  & \videolink{https://hopth.ru/21/b=prq49+s=16005+t=39+v=video+w=specimen}             \\
\cellcolor[HTML]{DA8BC3}{\morphtext{g}} & \descript{Initial group growth is almost entirely horizontal, with groups usually taking up only one row of cells. However, after an apparent timing cue groups perform a brief bout of aggressive vertical growth.} & \includesnapshot{\detokenize{snapshots/a=kin-group-id+idx=0+proc=0+series=16005+stint=45+thread=0+update=12991+_endeavor=16+_repro=0W10szaxUWV60t5Q+_slurm_job_id=25092156+_source=819521e-dirty+_treatment=bucket%prq49~diversity%0.50_series~mut_freq%1.00~mut_sever%1.00+ext=.png}}                  & \videolink{https://hopth.ru/21/b=prq49+s=16005+t=45+v=video+w=specimen}              \\
\cellcolor[HTML]{8C8C8C}{\morphtext{h}} & \descript{Groups grow horizontally and then proliferate vertically on a timing cue like morph $e$. However, after that timing cue cell proliferation is incessant with almost no resource retention.} & \includesnapshot{\detokenize{snapshots/a=kin-group-id+idx=0+proc=0+series=16005+stint=59+thread=0+update=11839+_endeavor=16+_repro=PrKgu7JjxV9bJooq+_slurm_job_id=25092244+_source=819521e-dirty+_treatment=bucket%prq49~diversity%0.50_series~mut_freq%1.00~mut_sever%1.00+ext=.png}}                  & \videolink{https://hopth.ru/21/b=prq49+s=16005+t=59+v=video+w=specimen}              \\
\cellcolor[HTML]{CCB974}{\morphtext{i}} & \descript{Irregular groups of mostly less than ten cells. Incessant proliferation with almost no resource retention leads to rapid group turnover.} & \includesnapshot{\detokenize{snapshots/a=kin-group-id+idx=0+proc=0+series=16005+stint=74+thread=0+update=12991+_endeavor=16+_repro=r5bt4vkTWWZvyaKe+_slurm_job_id=25092628+_source=819521e-dirty+_treatment=bucket%prq49~diversity%0.50_series~mut_freq%1.00~mut_sever%1.00+ext=.png}}                  & \videolink{https://hopth.ru/21/b=prq49+s=16005+t=74+v=video+w=specimen}              \\
\cellcolor[HTML]{64B5CD}{\morphtext{j}} & \descript{
Groups grow horizontally and then proliferate vertically on a timing cue like morph $e$. However, several viable horizontal-bar offspring groups form before force-fragementation.} & \includesnapshot{\detokenize{snapshots/a=kin-group-id+idx=0+proc=0+series=16005+stint=100+thread=0+update=8767+_endeavor=16+_repro=wSdyWubKcpinRb7H+_slurm_job_id=24415753+_source=8eb7a4e-dirty+_treatment=bucket%prq49~diversity%0.50_series~mut_freq%1.00~mut_sever%1.00+ext=.png}}                  & \videolink{https://hopth.ru/21/b=prq49+s=16005+t=100+v=video+w=specimen}
\end{tabular}

\caption{
Qualitative morph phenotype categorizations.
Color coding of morph IDs has no significance beyond guiding the eye in scatter plots where points are labeled by morph.
Snapshot visualizes spatial layout of kin groups on toroidal grid at a fixed point in time.
Each cell corresponds to a small square tile.
Color hue denotes and black borders divide outermost kin groups while color saturation denotes and white borders divide innermost kin groups.
}
\label{tab:morph_descriptions}

\end{table*}
}
