\begin{abstract}
Biotic selection, the emergent interplay between evolutionary processes and environmental effects driven by evolved traits, is broadly considered essential to the diversity and sophistication characteristic of biological life.
For this reason, substantial interest exists as to how changes in novelty, complexity, and adaptation unfold along lineages under selection driven by freeform, localized biotic interactions.
%  multidimensional, localized
% freeform, arbitrary, open to _______?
% it remains unclear to what
In particular, fundamental questions include whether novelty, complexity, and adaptation arise gradually versus episodically, and the extent to which they are coupled.
% extent to which these phenomena are coupled, and whether they spike in discrete events or are more gradual.
% In conjunction with , Experimental approaches to classical work concerning hypothetical plausibility of isolated scenarios, attempting to reproduce ETIs under controlled conditions tests the sufficiency of an integrated model to account for the origins of biological structure and complexity in concrete instances.
To explore these topics, we examine a detailed evolutionary history encompassing two sequential morphological novelties, drawn from a digital multicell model incorporating extensive biotic interactions.
% an \textit{in silico} model system competition for space and resources among multicell groups.
% open to arbitrary biotic interactions among independent replicator cells capable of arbitrary behaviors.
Using knockout experiments and competition trials, we trace quantitative measures of fitness, phenotypic complexity, and genetic complexity across a series of 100 intermediate snapshots captured across evolutionary time.
%  through knockout experiments and fitness changes through competition trials, we plot out a detailed evolutionary context timeline surrounding these novelties across series of 100 snapshot specimens. % captured across evolutionary time.
% We two phenotypic novelties in  arise — we see first cell groups adopt asymmetrical patterning; subsequently, we see these groups add a later secondary coordinated burst of outwards growth.
We find the establishment of an asymmetrical cigar patterning among multicells to be accompanied by moderate increases in both genetic complexity and fitness, as well as a substantial spike in phenotypic complexity.
However, later introduction of coordinated bursts in outward growth is accompanied by none of these effects.
% In a subsequent episode, a genome expansion drove strong fitness increase, but no notable changes in morphology or any complexity measures.
% Furthermore, strongest fitness effects are associated with no changes in genetic or phenotypic complexity --- although a notable later adaptive increase in genome size is observed.
Confirming the presence of biotic selection, we find some competition outcomes to reverse based on background presence of other evolved lineages.
Surprisingly, though, in one case we observe this effect associated with a sharp decrease in complexity and a loss of multicell patterning.
% Furthermore, late in evolutionary history, we see a sudden increase in genome size with strong fitness advantage but no apparent changes in complexity or life history.
% In several sampled, apparent "dead-end" lineages, we see apparent reversions with and loss of cooperative traits.
% Our a concrete
Finally, in partial contrast adaptationist framings of biological complexity, we do not find increases in genetic complexity to be directly associated with fitness gains.
Instead, increases in genetic complexity are significantly associated with future fitness gains, possibly indicative of a potentiating relationship.
% with complexity increase not significantly associated with fitness benefit, but
On the other hand, our observation that complexity measures sometimes --- albeit inconsistently --- spike coincident with morphological novelty partially supports adaptationist framings of complexity.
% Providing a detailed, concrete glimpse into the evolution of complexity, novelty, and adaptation under a regime influenced by biotic selection,
% and provide a foundation for future work in tandem with new avenues of experimental evolution in \textit{in vivo},
% foundation for future work with replay experiments and multistrain tests across independent evolutionary trajectories

% Abstract length should not exceed 250 words
% Continuing generation of novelty, complexity, and adaptation are well-established as core aspects of open-ended evolution.
% However, it has yet to be firmly established to what extent these phenomena are coupled and by what means they interact.
% In this work, we track the co-evolution of novelty, complexity, and adaptation in a case study from the DISHTINY simulation system, which is designed to study the evolution of digital multicellularity.
% In this case study, we describe ten qualitatively distinct multicellular morphologies, several of which exhibit asymmetrical growth and distinct life stages.
% We contextualize the evolutionary history of these morphologies with measurements of complexity and adaptation.
% Our case study suggests a loose --- sometimes divergent --- relationship can exist among novelty, complexity, and adaptation.
\end{abstract}
