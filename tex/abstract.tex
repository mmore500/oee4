\begin{abstract}
Biotic selection, emergent feedback from evolved traits in shaping the environmental context of evoluting populations, is broadly understood as a key driver of diversity and sophistication characteristic of biological life.
For this reason, great interest exists in how complexity develops along lineages under selection driven by biotic interactions.
%  multidimensional, localized
% freeform, arbitrary, open to _______?
% it remains unclear to what
Fundamental questions include whether complexity arises gradually versus episodically, and the extent to which it is coupled with novelty and adaptation.
% extent to which these phenomena are coupled, and whether they spike in discrete events or are more gradual.
% In conjunction with , Experimental approaches to classical work concerning hypothetical plausibility of isolated scenarios, attempting to reproduce ETIs under controlled conditions tests the sufficiency of an integrated model to account for the origins of biological structure and complexity in concrete instances.
To explore these topics, we examine a detailed evolutionary history encompassing a sequential pair of morphological novelties, drawn from a digital multicell model incorporating extensive freeform, localized biotic interactions.
% an \textit{in silico} model system competition for space and resources among multicell groups.
% open to arbitrary biotic interactions among independent replicator cells capable of arbitrary behaviors.
Using knockout experiments and competition assays, we trace quantitative measures of fitness, phenotypic complexity, and genetic complexity across a series of 100 intermediate snapshots captured across this evolutionary history.
%  through knockout experiments and fitness changes through competition trials, we plot out a detailed evolutionary context timeline surrounding these novelties across a series of 100 snapshot specimens. % captured across evolutionary time.
% We two phenotypic novelties in  arise — we see first cell groups adopt asymmetrical patterning; subsequently, we see these groups add a later secondary coordinated burst of outwards growth.

% However, a later introduction of coordinated bursts in outward growth is accompanied by none of these effects.
% In a subsequent episode, a genome expansion drove strong fitness increase, but no notable changes in morphology or any complexity measures.
% Furthermore, strongest fitness effects are associated with no changes in genetic or phenotypic complexity --- although a notable later adaptive increase in genome size is observed.
% Furthermore, late in evolutionary history, we see a sudden increase in genome size with strong fitness advantage but no apparent changes in complexity or life history.
% In several sampled, apparent "dead-end" lineages, we see apparent reversions with and loss of cooperative traits.
In contrast to neutral framings of biological complexity, we observe instances where sharp spikes in complexity coincide with morphological novelties --- for instance, the initial establishment of patterning among multicells accompanied a near tripling in one functional complexity measure.
Contrasting adaptationist framings of biological complexity, though, we do not find an association between functional complexity and fitness gains.
Instead, we find connections between genome expansion and fitness gains, in a manner suggestive of biotic selection.
Grounded by direct observation of evolution in a concrete instance, these findings underscore that constructive outcomes of evolution are not necessarily monolithic --- but can instead arise from loose interactions among complexity, novelty, and adaptation.
% Abstract length should not exceed 250 words
% Continuing generation of novelty, complexity, and adaptation is well-established as a core aspect of open-ended evolution.
% However, it has yet to be firmly established to what extent these phenomena are coupled and by what means they interact.
% In this work, we track the co-evolution of novelty, complexity, and adaptation in a case study from the DISHTINY simulation system, which is designed to study the evolution of digital multicellularity.
% In this case study, we describe ten qualitatively distinct multicellular morphologies, several of which exhibit asymmetrical growth and distinct life stages.
% We contextualize the evolutionary history of these morphologies with measurements of complexity and adaptation.
% Our case study suggests a loose --- sometimes divergent --- relationship can exist among novelty, complexity, and adaptation.
\end{abstract}
