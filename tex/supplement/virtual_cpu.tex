\section{Virtual CPU} \label{sec:virtual_cpu}

Each cardinal hosts a signalgp-lite virtual CPU.
Each CPU can host up to 16 active virtual cores.
If additional cores are required after all 16 available are in use, the oldest active core is killed and replaced.
Each virtual core contains 8 virtual \texttt{float} registers.

Cores execute round-robin in quasi-parallel, with up to 8 instructions being executed on a single core before execution shifts to the next active core.

Like SignalGP, the signalgp-lite system uses tag-matching to determine which modules to activate in response to incoming signals from the environment, from other agents (i.e., messages), and from internal events (i.e., execution of \texttt{call} and \texttt{fork} instructions).

In the DISHTINY simulation, each CPU hosts two independent module-lookup data structures.
The first module-lookup data structure is used to activate modules in response to internally-generated signals, messages from other cardinals within the same cell, and environmental events;
this module-lookup data structure contains \textit{all} modules within the genetic program.
The second module-lookup data structure is used to activate modules in response to messages from other cells;
this module-lookup data structure contains only modules with bitstring tags that end in 0.
(Hence, the subset of modules with bitstring tags that end in 1 \textit{cannot} be activated by messages from other cells, so that sensitive functionality like resource sharing and apoptosis can be protected from potentially malicious exploitation.)

We also use a tag-matching system to route \texttt{jump} instructions executed within a module.
When a module is loaded, all \texttt{local anchor} instructions are registered within a tag-matching data structure.
The local \texttt{jump} instruction routes to the best-matching local anchor.
If no matching local anchor is available, then no jump is performed and execution continues as if a \texttt{nop} instruction had elapsed.
