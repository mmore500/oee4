\section{Program Generation and Mutation}

Initial populations were seeded with programs consisting of 128 randomly generated instructions.
Program length was capped at 4096 instructions.

Mutation was applied to one in 10 reproductions where any kin group commonality was maintained and to 10 in 10 reproductions where it was not.
If mutation occurred, bits in the binary representation of the genome were flipped with 0.02\% probability.
If mutation occurred, sequence mutations were also introduced into the program at a per-site rate of 0.1\%.
Half of sequence mutations were deletion events, with a number of sites deleted drawn uniformly between 0 and 8.
Half of sequence mutations were insertion events, with a number of sites inserted drawn uniformly between 0 and 8.
When sites were inserted, half of the time randomly-generated instructions were added and half of the time the preceding sequence of instructions was duplicated.
With 0.1\% probability a sequence mutation took on severe intensity, meaning that the number of sites inserted or deleted was drawn uniformly between 0 and program size rather than between 0 and 8.
