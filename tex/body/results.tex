\section{Results}

\subsection{Evolutionary History}

% https://mybinder.org/v2/gh/mmore500/dishtiny/e17e6d5e258b7aacac72d44922008ab14e80e182?filepath=binder%2Fbucket%3Dprq49%2Fa%3Dall_stints_all_series_profiles%2Bendeavor%3D16%2Fcase_study_16005.ipynb
Due to the distributed nature of the experimental framework, we did not perform perfect phylogeny tracking.
However, we did track the total number of ancestors seeded into stint 0 with extant descendants.
At the end of stints 0 and 1, three distinct original phylogenetic roots were present in the population.
From stint 2 onward, only two distinct original phylogenetic roots were present.

% https://mybinder.org/v2/gh/mmore500/dishtiny/e17e6d5e258b7aacac72d44922008ab14e80e182?filepath=binder%2Fbucket%3Dprq49%2Fa%3Dall_stints_all_series_profiles%2Bendeavor%3D16%2Fcase_study_16005.ipynb
We performed follow-up analyses on specimens sampled from the lowest original phylogenetic root ID present in the population.
For the first two stints, this was root ID 2,378.
During stint 2, original phylogenetic root 2,378 went extinct.
So, all further follow-up analyses were sampled from descendants of ancestor 12,634.

We also tracked the number of genomes reconstituted at the outset of each stint with extant descendants at the end of that stint.
This count grows from approximately 10 around stint 15 to upwards of 30 around stint 40 (Supplementary Figure \ref{fig:phylogeny:stint_roots}\citep{Moreno_2021}).
Among descendants of the lowest original phylogenetic root, the number of independent lineages spanning a stint also increases from around 5 to around 15
(Supplementary Figure \ref{fig:phylogeny:lowestroot_stint_roots} \citep{Moreno_2021}).
This decrease in phylogenetic consolidation on a stint-by-stint basis correlates with the waning number of simulation updates performed per stint (Supplementary Figures \ref{fig:phylogeny:updates_vs_stint_roots} and \ref{fig:phylogeny:log_updates_vs_stint_roots} \citep{Moreno_2021}).
More complete phylogenetic data will be necessary in future experiments to address questions about the possibility of long-term stable coexistence beyond the two strains supported under the explicit     diversity maintenance scheme.

% https://mybinder.org/v2/gh/mmore500/dishtiny/e17e6d5e258b7aacac72d44922008ab14e80e182?filepath=binder%2Fbucket%3Dprq49%2Fa%3Dall_stints_all_series_profiles%2Bendeavor%3D16%2Fcase_study_16005.ipynb
On the specimen from stint 100 used in the final case study, an evolutionary history of 20,212 cell generations had elapsed.
Of these cellular reproductions, 11,713 (58\%) had full kin group commonality, 7,174 had partial kin group commonality (35\%), and 1,325 had no kin group commonality (7\%).
On this specimen, 1,672 mutation events had elapsed.
During these events, 7,240 insertion-deletion alterations had occurred and 26,153 point mutations had occurred.
This strain experienced a selection pressure of 18\% over its evolutionary history, meaning that only 82\% of the mutations that would be expected given the number of cellular reproductions that had elapsed were present.

\begin{figure*}
\centering
\includegraphics[width=0.8\linewidth]{{submodule/dishtiny_event_tag_phylogenetics/teeplots/rerooted_parsimony_tree_no_outliers/viz=draw+ext=}}

\caption{
Phylogeny of sampled focal strain representatives across stints reconstructed using parsimony algorithm \citep{cock2009biopython}.
Each leaf node corresponds to a sampled representative.
Representatives from stints 0 and 1, which share no common ancestry with representatives from other stints, are excluded.
Numbers refer to stint that each representative was sampled from.
Color coding and parentheticals of stint labels correspond to qualitative morph codes described in Table \ref{tab:morph_descriptions}.
}
\label{fig:phylo_parsimony_tree}
\end{figure*}


% https://github.com/mmore500/dishtiny_event_tag_phylogenetics/commit/ecc2ab8a580b125bea9f661e49cd4d7f34bfa1bb
In order to characterize the evolutionary history of the experiment in greater detail, we performed a parsimony-based phylogenetic reconstruction on the sampled representative specimens from each stint, shown in Figure \ref{fig:phylo_parsimony_tree}.
We used genomes' fixed-length blocks of 35 64-bit tags that mediate environmental interactions as the basis for this reconstruction.
These tag blocks underwent bitwise mutation over the course of the experiment.\footnote{
In future experiments, we plan to incorporate new methodology for ``Hereditary Stratigraph'' genome annotations expressly designed to facilitate phylogenetic reconstruction \citep{moreno2022hereditary}.
}
Supplementary Figure \ref{fig:phylo_distance_matrix_heatmap} shows hamming distance between all pairs of tag blocks.
We additionally tried several other tree inference methods, discussed in supplementary material; however, these yielded lower-quality reconstructions.

Although the phylogeny of stint representatives includes many instances breaking strict stint-on-stint ladderization (i.e., each stint's representative descending directly from the preceding stint's representative), we did not observe evidence of long-term coexistence of clades over more than ten stints.

\subsection{Qualitative Morphological Categorizations}

\input{lib/dissertationexclude.tex}
\input{lib/dissertationonly.tex}

\newcommand{\includesnapshot}[1] {%
\adjustbox{trim={0.05\width} {0.35\width} {0.05\width} {0.35\width},clip}%
    {\includegraphics[height=\dissertationexclude{0.3}\dissertationonly{0.2}\textheight]{#1}}
}
\newcommand{\morphtext}[1] {%
\color[HTML]{FFFFFF} \huge \raisebox{1.2em}{\textbf{#1}}%
}
\newcommand{\videolink}[1] {%
\raisebox{
\dissertationexclude{2.8em}
\dissertationonly{2.2em}
}{\begin{minipage}{1.3 cm} \tiny\url{#1} \end{minipage}}
}
\newcommand{\descript}[1] {%
\raisebox{
  \dissertationexclude{2.8em}
  \dissertationonly{2.2em}
}{\begin{minipage}{\linewidth}\dissertationonly{\tiny} #1 \end{minipage}}
}


{
\catcode`\%=12
\begin{table*}
\begin{tabular}{cp{0.4\textwidth}ll}
\multicolumn{1}{l}{\textbf{ID}}               & \textbf{Morphology} & \textbf{Snapshot} & \textbf{Video} \\
\cellcolor[HTML]{4C72B0}{ \morphtext{a} } & \descript{Individual cells, no multi-cellular kin groups. Resource use is low---most cells simply hoard resource until their stockpile is beyond sufficient to reproduce. Only a handfuls of cells intermittently expend resource.} & \includesnapshot{\detokenize{snapshots/a=kin-group-id+idx=0+proc=0+series=16005+stint=0+thread=0+update=28991+_endeavor=16+_repro=rvV3h5Ru0tlNBvox+_slurm_job_id=24414678+_source=8eb7a4e-dirty+_treatment=bucket%prq49~diversity%0.50_series~mut_freq%1.00~mut_sever%1.00+ext=}}             &  \videolink{https://hopth.ru/21/b=prq49+s=16005+t=0+v=video+w=specimen}          \\
\cellcolor[HTML]{DD8452}{\morphtext{b}} & \descript{Mostly individual cells, with some two-, three-, and four-cell groups evenly spread out. Resource usage occurs in short spurts in one or two adjacent cells. } & \includesnapshot{\detokenize{snapshots/a=kin-group-id+idx=0+proc=0+series=16005+stint=1+thread=0+update=14271+_endeavor=16+_repro=hVlcCQPvlFIR0ckX+_slurm_job_id=25050103+_source=819521e-dirty+_treatment=bucket%prq49~diversity%0.50_series~mut_freq%1.00~mut_sever%1.00+ext=}}                 & \videolink{https://hopth.ru/21/b=prq49+s=16005+t=1+v=video+w=specimen}              \\
\cellcolor[HTML]{55A868}{\morphtext{c}} & \descript{Large multi-cellular groups dominate, consisting of hundreds of cells. Group growth is unchecked and continues until cells' resource stockpiles are entirely depleted by the excess group size penalty.} & \includesnapshot{\detokenize{snapshots/a=kin-group-id+idx=0+proc=0+series=16005+stint=2+thread=0+update=17471+_endeavor=16+_repro=fHVOHaP3ZgZwIfuP+_slurm_job_id=25050099+_source=819521e-dirty+_treatment=bucket%prq49~diversity%0.50_series~mut_freq%1.00~mut_sever%1.00+ext=}}                  & \videolink{https://hopth.ru/21/b=prq49+s=16005+t=2+v=video+w=specimen}             \\
\cellcolor[HTML]{C44E52}{\morphtext{d}} & \descript{Clear groups of 10 to 15 cells in size form. Cell proliferation appears somewhat more active at the periphery of groups compared to the interior.} & \includesnapshot{\detokenize{snapshots/a=kin-group-id+idx=0+proc=0+series=16005+stint=14+thread=0+update=16959+_endeavor=16+_repro=BnqHEceSUrehEdlE+_slurm_job_id=25086959+_source=819521e-dirty+_treatment=bucket%prq49~diversity%0.50_series~mut_freq%1.00~mut_sever%1.00+ext=}}                  & \videolink{https://hopth.ru/21/b=prq49+s=16005+t=14+v=video+w=specimen}               \\
\cellcolor[HTML]{8172B3}{\morphtext{e}} & \descript{Groups are visibly elongated along the horizontal axis. After initial development, some gradual, irregular growth occurs along the vertical axis.} & \includesnapshot{\detokenize{snapshots/a=kin-group-id+idx=0+proc=0+series=16005+stint=15+thread=0+update=16639+_endeavor=16+_repro=HJOcggiTrmZJVbXg+_slurm_job_id=25086417+_source=819521e-dirty+_treatment=bucket%prq49~diversity%0.50_series~mut_freq%1.00~mut_sever%1.00+ext=}}                  & \videolink{https://hopth.ru/21/b=prq49+s=16005+t=15+v=video+w=specimen}              \\
\cellcolor[HTML]{937860}{\morphtext{f}} & \descript{Groups are horizontally elongated similarly to morphology $e$, but have a greater consistent vertical thickness of three or four cells.} & \includesnapshot{\detokenize{snapshots/a=kin-group-id+idx=0+proc=0+series=16005+stint=39+thread=0+update=10303+_endeavor=16+_repro=kk8MYN0ff7wGODOC+_slurm_job_id=25091828+_source=819521e-dirty+_treatment=bucket%prq49~diversity%0.50_series~mut_freq%1.00~mut_sever%1.00+ext=}}                  & \videolink{https://hopth.ru/21/b=prq49+s=16005+t=39+v=video+w=specimen}             \\
\cellcolor[HTML]{DA8BC3}{\morphtext{g}} & \descript{Initial group growth is almost entirely horizontal, with groups usually taking up only one row of cells. However, after an apparent timing cue groups perform a brief bout of aggressive vertical growth.} & \includesnapshot{\detokenize{snapshots/a=kin-group-id+idx=0+proc=0+series=16005+stint=45+thread=0+update=12991+_endeavor=16+_repro=0W10szaxUWV60t5Q+_slurm_job_id=25092156+_source=819521e-dirty+_treatment=bucket%prq49~diversity%0.50_series~mut_freq%1.00~mut_sever%1.00+ext=}}                  & \videolink{https://hopth.ru/21/b=prq49+s=16005+t=45+v=video+w=specimen}              \\
\cellcolor[HTML]{8C8C8C}{\morphtext{h}} & \descript{Groups grow horizontally and then proliferate vertically on a timing cue like morph $e$. However, after that timing cue cell proliferation is incessant with almost no resource retention.} & \includesnapshot{\detokenize{snapshots/a=kin-group-id+idx=0+proc=0+series=16005+stint=59+thread=0+update=11839+_endeavor=16+_repro=PrKgu7JjxV9bJooq+_slurm_job_id=25092244+_source=819521e-dirty+_treatment=bucket%prq49~diversity%0.50_series~mut_freq%1.00~mut_sever%1.00+ext=}}                  & \videolink{https://hopth.ru/21/b=prq49+s=16005+t=59+v=video+w=specimen}              \\
\cellcolor[HTML]{CCB974}{\morphtext{i}} & \descript{Irregular groups of mostly less than ten cells. Incessant proliferation with almost no resource retention leads to rapid group turnover.} & \includesnapshot{\detokenize{snapshots/a=kin-group-id+idx=0+proc=0+series=16005+stint=74+thread=0+update=12991+_endeavor=16+_repro=r5bt4vkTWWZvyaKe+_slurm_job_id=25092628+_source=819521e-dirty+_treatment=bucket%prq49~diversity%0.50_series~mut_freq%1.00~mut_sever%1.00+ext=}}                  & \videolink{https://hopth.ru/21/b=prq49+s=16005+t=74+v=video+w=specimen}              \\
\cellcolor[HTML]{64B5CD}{\morphtext{j}} & \descript{
Groups grow horizontally and then proliferate vertically on a timing cue like morph $e$. However, several viable horizontal-bar offspring groups form before force-fragementation.} & \includesnapshot{\detokenize{snapshots/a=kin-group-id+idx=0+proc=0+series=16005+stint=100+thread=0+update=8767+_endeavor=16+_repro=wSdyWubKcpinRb7H+_slurm_job_id=24415753+_source=8eb7a4e-dirty+_treatment=bucket%prq49~diversity%0.50_series~mut_freq%1.00~mut_sever%1.00+ext=}}                  & \videolink{https://hopth.ru/21/b=prq49+s=16005+t=100+v=video+w=specimen}
\end{tabular}

\caption{
Qualitative morph phenotype categorizations.
Color coding of morph IDs has no significance beyond guiding the eye in scatter plots where points are labeled by morph.
Snapshot visualizes spatial layout of kin groups on toroidal grid at a fixed point in time.
Each cell corresponds to a small square tile.
Color hue denotes and black borders divide outermost kin groups while color saturation denotes and white borders divide innermost kin groups.
}
\label{tab:morph_descriptions}

\end{table*}
}


We performed a qualitative survey of the evolved life histories along the evolutionary timeline by analyzing video recordings of monocultures of each stint's representative specimen.

Table \ref{tab:morph_descriptions} summarizes the ten morphological categories we grouped specimens into.
In brief, specimens from early stints largely grew as unicellular or small multicellular groups (morphs $a$, $b$).
Then, the specimen from stint 14 grew as larger, symmetrical groups (morph $d$).
At stint 15, a distinct, asymmetrical horizontal bar morphology evolved (morph $e$).
%todo structure above is confusing
% Consistently left/right, indicating that somehow broke symmetrical of the simulation.
At stint 45, a delayed secondary spurt of group growth in the vertical direction arose (morph $g$).
This morphology was sampled frequently until stint 60 when morph $e$ began to be sampled primarily again.
However, morph $g$ was observed as late as stint 90.

Phylogenetic analysis (Figure \ref{fig:phylo_parsimony_tree}) indicates that observations of morph $e$ at stint 53 and onward are instances of secondary loss rather than retention of trait $e$ by a separate lineage coexisting with the lineage expressing morph $g$.
Three separate reversion events from morph $g$ to morph $e$ appear likely.
Interestingly, morph $g$ individuals at stints 89 and 90 appear to represent subsequent trait re-gain after reversion from morph $g$ to morph $e$.

Table \ref{tab:morph_descriptions} provides more detailed descriptions of each qualitative morph category as well as video and a still image example of each.
Supplementary Table \ref{tab:morph_by_stint} provides morph categorization for each stint as well as links to view the stint's specimen in a video or in-browser web simulation \citep{Moreno_2021}.

% \subsection{Multicellular Phenotypic Traits}

% In order for a transition in individuality, cells have to diminish their own immediate reproductive interests in order to prioritize the interests of the collective.
% In previous work with an earlier version of the system, we've seen multicellular traits evolve such as (blah from abstract) %todo
% \citep{moreno2021exploring}.

% The formation of kin group morphologies in this system does not necessarily imply a transition in individuality.
% It is necessary to test to see if cells within groups are meaningfully cooperating.

% Because cell reproduction is necessarily adversarial in this system, we can test to see whether cells preferentially target or spare their group members.
% A conflict ratio of less than 1 implies that cells preferentially spare their members.
% We observed a  0.5 most of the time.

% Supplementary Figure \ref{fig:conflict:exactly_one} shows the  . %todo
% Supplementary Figure \ref{fig:conflict:at_least_one}
% Supplementary Figure \ref{fig:conflict:exactly_two}.
% However, could be an artifact of cell-offspring cooperation.

% Interestingly, this strain appears to be using ``Is Child Cell Of,'' ``Is Parent Cell Of,'' ``Is Child Group Of,'' and the raw kin group ID of neighbors (but not the raw kin group ID of self).

% %todo change wording, not sure what you're hoping to say with first phrase
% To more conclusively, in future work we plan to look at the reproductive output at the interior of multicellular groups to see if these cells diminish their own reproductive output and to control for cellular relatedness when calculating this element of group cooperation.

% We analyzed other traits characteristic of multicellularity, as well.

% We did not see widespread resource sharing over the evolutionary history.
% Only tiny fractions of cells, less than 1\%, appeared to share tiny amounts of the resource, less than 1\% of the resource required to reproduce per update (Supplementary Figures \ref{fig:resource_sharing:fraction_sharing_monoculture} and \ref{fig:resource_sharing:sharing_amount_monoculture}).
% However, in ten strains sampled along the evolutionary lineage, context-dependent resource-sending behavior contributed to fitness (Supplementary Figure \ref{fig:writable_perturbation}).
% Interestingly, another strain present in the population that was not the subject of analysis appeared to evolve widespread, higher-throughput resource sharing after stint 60 (Supplementary Figures \ref{fig:resource_sharing:fraction_sharing_evolve} and \ref{fig:resource_sharing:sharing_amount_evolve}).

% We observed widespread apoptosis over the evolutionary history.
% As shown in Supplementary Figure \ref{fig:apoptosis:apoptosis_stint}, in most monocultured representative strains, between 10 and 30\% of cell deaths were due to apoptosis.
% However, it is not clear if this apoptosis returned a fitness benefit.
% No context-dependent apoptosis behavior contributed to fitness for any sampled strain over the evolutionary history (Supplementary Figure \ref{fig:writable_perturbation}).
% It is unclear whether context-independent apoptosis might have a fitness benefit.

% The behavior seems like cells from within a group don't explicitly recognize one another according to group membership. However, they do recognize their cell parent/child and arrange themselves in a somewhat linear fashion that minimizes the potential for conflict between cells.
% Overall, there are elements of cooperation but is unclear to what extent this particular strain constitutes a proper fraternal transition in individuality.

\subsection{Fitness}

\begin{sidewaysfigure*}
\thisfloatpagestyle{mylandscape}%
\rotatesidewayslabel%
\centering
\includegraphics[width=\linewidth]{{submodule/dishtiny/binder/bucket=prq49/a=adaptation_assays+endeavor=16/teeplots/hue=num-updates-elapsed+viz=facet-heatmap+x=biotic-background+y=competition-stint+ext=}}

\caption{
Number updates elapsed during fixed-duration adaptation assay competitions for sampled representative specimen (top) and population-level adaptation (bottom).
See Figure \ref{fig:adaptation_assay_cartoon} for explanation of competition biotic backgrounds.
See Supplementary Figure \ref{fig:num_updates_elapsed_barplot} for confidence interval estimates of mean updates elapsed during competition expedriments and Supplementary Figure \ref{fig:num_updates_elapsed_boxplot} for distributions of updates elapsed during competition experiments.
}
\label{fig:num_updates_elapsed_heatmap}
\end{sidewaysfigure*}

\begin{sidewaysfigure*}
\centering
\includegraphics[width=\linewidth]{{submodule/dishtiny/binder/bucket=prq49/a=adaptation_assays+endeavor=16/teeplots/hue=biotic-background+stint=1-50+viz=facet-boxplot+x=competition-stint+y=update+ext=}}
\includegraphics[width=\linewidth]{{submodule/dishtiny/binder/bucket=prq49/a=adaptation_assays+endeavor=16/teeplots/hue=biotic-background+stint=51-100+viz=facet-boxplot+x=competition-stint+y=update+ext=}}

\caption{
Number updates elapsed during fixed-duration adaptation assay competitions for sampled representative specimen (upper panels) population-level adaptation (lower panels).
Figure is split into two rows due to layout considerations.
See Figure \ref{fig:adaptation_assay_cartoon} for explanation of competition biotic backgrounds.
}
\label{fig:num_updates_elapsed_boxplot}
\end{sidewaysfigure*}

\input{fig/adaptation/num_updates_elapsed_barplot.tex}
\begin{sidewaysfigure*}
\centering
\includegraphics[width=\linewidth]{{submodule/dishtiny/binder/bucket=prq49/a=adaptation_assays+endeavor=16/teeplots/hue=mean-competition-prevalence+viz=facet-heatmap+x=biotic-background+y=competition-stint+ext=}}

\caption{
TODO
}
\label{fig:mean_competition_prevalence}
\end{sidewaysfigure*}

\begin{sidewaysfigure*}
\centering
\includegraphics[width=\linewidth]{{submodule/dishtiny/binder/bucket=prq49/a=adaptation_assays+endeavor=16/teeplots/hue=biotic-background+viz=facet-boxplot+x=competition-stint+y=focal-prevalence+ext=}}

\caption{
TODO
}
\label{fig:mean_competition_prevalence_boxplot}
\end{sidewaysfigure*}

\begin{sidewaysfigure*}
\centering

\includegraphics[width=\linewidth]{{submodule/dishtiny/binder/bucket=prq49/a=adaptation_assays+endeavor=16/teeplots/hue=biotic-background+stint=1-50+viz=facet-barplot+x=competition-stint+y=focal-prevalence+ext=}}

\includegraphics[width=\linewidth]{{submodule/dishtiny/binder/bucket=prq49/a=adaptation_assays+endeavor=16/teeplots/hue=biotic-background+stint=51-100+viz=facet-barplot+x=competition-stint+y=focal-prevalence+ext=}}

\caption{
TODO
}
\label{fig:mean_competition_prevalence_barplot}
\end{sidewaysfigure*}

\begin{figure*}
\centering

\begin{subfigure}{\textwidth}

\includegraphics[width=\linewidth]{{submodule/dishtiny/binder/bucket=prq49/a=adaptation_assays+endeavor=16/teeplots/hue=setup+viz=barplot+x=competition-stint+y=fitness-differential-focal+ext=}}
\caption{
Calculated fitness differential between competing strains based on population composition at the end of competition experiments.
Zero is neutral.
Error bars are 95\% confidence intervals.
}
\end{subfigure}

\begin{subfigure}{\textwidth}
\includegraphics[width=\linewidth]{{submodule/dishtiny/binder/bucket=prq49/a=adaptation_assays+endeavor=16/teeplots/hue=setup+viz=boxplot+x=competition-stint+y=focal-prevalence+ext=}}
\caption{
Fractional composition of focal population at the end of competition experiments.
A neutral outcome corresponds to even (0.5) composition, annotated with a horizontal line.
}
\end{subfigure}

\begin{subfigure}{\textwidth}
\includegraphics[width=\linewidth]{{submodule/dishtiny/binder/bucket=prq49/a=adaptation_assays+endeavor=16/teeplots/hue=setup+viz=countplot+x=competition-stint+ext=}}
\caption{
Number competitions out of 20 won by first strain.
Ten competitions won corresponds to a perfectly neutral outcome.
Eighteen and more or two or less competitions won were considered to indidicate a significant fitness difference between strains.
These thresholds for significance annotated with horizontal lines.
}
\end{subfigure}

\caption{
Control adaptation experiments for selected stints.
Control experiments were performed by competing two identical genomes or populations against each other with the contemporary biotic background, with the prefatory biotic background, or with no biotic background.
See Figure \ref{fig:adaptation_assay_cartoon} for summary of adaptation experiment design.
}
\label{fig:adaptation_control}
\end{figure*}

\begin{sidewaysfigure*}
\centering
\includegraphics[width=\linewidth]{{submodule/dishtiny/binder/bucket=prq49/a=adaptation_assays+endeavor=16/teeplots/hue=symlog-median-fitness-differential+viz=facet-heatmap+x=biotic-background+y=competition-stint+ext=}}

\caption{
Median calculated fitness differential outcomes of competition experiments.
Zero fitness differential corresponds to a neutral result, color mapped to white.
Blue indicates positive fitness differential (fitness gain) compared to the previous stint and red indicates negative fitness differential (fitness loss).
Color coding and parentheticals of stint labels correspond to qualitative morph codes described in Table \ref{tab:morph_descriptions}.
Note that color intensity is plotted on a symlog scale due to distribution of fitness differentials over multiple orders of magnitude.
Upper panels shows results for sampled focal strain genome, lower panel shows results for entire focal strain population.
See Figure \ref{fig:adaptation_assay_cartoon} for explanation of competition biotic backgrounds.
}
\label{fig:median_fitness_differential_symlog}
\end{sidewaysfigure*}


% \begin{figure}

\begin{subfigure}{0.5\textwidth}

\includegraphics[width=\linewidth]{{plots/fitness/bucket=prq49+cat=morph+endeavor=16+transform=filter-Series-16005+viz=letterscatter-vline+x=stint+y=mean-doubling-time-growth-rate+ext=}}

\end{subfigure}%
\begin{subfigure}{0.5\textwidth}

\includegraphics[width=\linewidth]{{plots/fitness/bucket=prq49+cat=morph+endeavor=16+transform=filter-Series-16005+viz=letterscatter-vline-hline+x=stint+y=median-fitness-differential-against-immediate-predecessor-population+ext=}}

\caption{Magnitude of fitness differential against immediately-preceding stint population.
Positive fitness differential indicates greater fitness compared to predecessor.
Solid horizontal line indicates neutral fitness differential.
}
\label{fig:fitness:fitness_magnitude}

\end{subfigure}%

\begin{subfigure}{0.5\textwidth}

\includegraphics[width=\linewidth]{{plots/fitness/bucket=prq49+cat=morph+endeavor=16+transform=filter-Series-16005+viz=letterscatter-vline-binomialh0+x=stint+y=fraction-immediate-predecessor-battles-won+ext=}}

\caption{
Fraction of 20 independent competitions that were won against immediate predecessor population.
Blue horiziontal lines represent significance level $p < 0.05$ for binomial null hypothesis.
Neutral outcomes fall inside the blue bars, significant fitness increases fall above them, and significant fitness decreases fall below them.
}
\label{fig:fitness:fitness_neutrality}

\end{subfigure}%


\caption{ Fitness assays.
Color coding and letters correspond to qualitative morph codes described in Table \ref{tab:morph_descriptions}.
Dotted vertical line denotes emergence of morph $e$.
Dashed vertical line denotes emergence of morph $g$. }
\label{fig:fitness}

\end{figure}

In order to assess ongoing changes in fitness, we performed fitness competitions between the representative focal strain specimen sampled at each stint and the focal strain population from the preceding stint.
(Recall from Section \ref{sec:methods;ch:measuring-cna} that, due to a diversity maintenance procedure, two completely independent strains coexisted over the course of the experiment --- the ``focal'' strain selected for analysis and a ``background'' strain.)
Using the population, rather than the representative specimen, from the preceding stint as the competitive baseline ensured more focused, consistent measurement of the fitness properties of the specimen at the current stint (e.g., preventing skewed results from a sampled ``dud'' at the preceding stint).

We performed 20 independent replicates of each competition.
Competing strains were well-mixed within the full-sized toroidal grid at the outset of each competition, which lasted for 10 minutes of wall time.
This was sufficient to simulate about 8,000 updates at stint 0 and 2,000 updates at stint 100 (Supplementary Figures \ref{fig:num_udpates_elapsed_heatmap}, \ref{fig:num_updates_elapsed_barplot}, and \ref{fig:num_updates_elapsed_boxplot}).
We determined that a gain of fitness had occurred if the current stint specimen constituted a population majority at the conclusion of more than 17 of those competitions, corresponding to a significance level of $p < 0.005$ under the two-tailed binomial null hypothesis.
Likewise, we deemed winning fewer than 3 competitions a significant fitness loss.

\begin{figure*}

\includegraphics[width=\linewidth]{{submodule/dishtiny/binder/bucket=prq49/a=adaptation_assays+endeavor=16/teeplots/col=biotic-background+kind=count+row=assay-subject+viz=barlabel-catplot+x=outcome+ext=}}

\caption{
Distributions of adaptation assay outcomes over all stints.
For each adaptation assay, three outcomes were possible: significant fitness gain, significant fitness loss, or no significant fitness change (``neutral'').
Significance cutoff $p < 0.005$ was used.
A fitness loss --- color coded red --- corresponds to winning 2 or fewer competitions out of 20 against the preceding stint's focal strain population.
A fitness gain --- color coded green --- corresponds to winning 18 or more competitions out of 20 against the preceding stint's focal strain population.
Neutral fitness outcomes are color coded yellow.
Outcome counts are accumulated over experiments from stint 1 through stint 100.
Upper row shows results for sampled focal strain genome, lower row shows results for entire focal strain population.
See Figure \ref{fig:adaptation_assay_cartoon} for explanation of competition biotic backgrounds.
See Figure \ref{fig:outcome_count_joint_distns} for joint distributions of fitness outcomes across biotic backgrounds.
}
\label{fig:outcome_count_distns}
\end{figure*}


Of the 100 competition assays performed, 57 indicated significant fitness gain, 23 were neutral, and 20 indicated significant fitness loss (shown in upper right of Figure \ref{fig:fitness:outcome_count_distns}, at the intersection of the ``Biotic Background, Without'' column and ``Assay Subject, Specimen'' row.)

Surprised by the frequency of deleterious outcomes, we performed a second set of experiments to investigate whether these outcomes could be explained as sampling of ``dud'' representatives.
In these competition assays, we competed the entire focal strain population against the focal strain population from the preceding stint.
However, we observed a similar result: 50 assays indicated significant fitness gain, 34 were neutral, and 16 indicated significant fitness loss (shown in lower right of Figure \ref{fig:fitness:outcome_count_distns}, at the intersection of the ``Biotic Background, Without'' column and ``Assay Subject, Population'' row.)

\begin{figure*}
\centering
\includegraphics[width=0.8\linewidth]{{img/adaptation-assay-cartoon}}

\caption{
Summary of adaptation assays used to measure adaptation of sampled representative specimen (top) population-level adaptation (bottom).
}
\label{fig:adaptation_assay_cartoon}
\end{figure*}


Next, we investigated whether the presence of the background strain as a ``biotic background'' influenced fitness.
We repeated the two experiments described above (specimen and population competition assays), but inserted the background strain as half of the initial well-mixed population.
In one assay setup, we used the background strain population from the current stint.
We refer to this as ``contemporary biotic background.''
In another, which we call ``prefatory biotic background,'' we used the background strain population from the previous stint.
We refer to the original competition assays absent the background strain as ``without biotic background.''
Figure \ref{fig:adaptation_assay_cartoon} summarizes these competition assay designs.

\begin{figure*}
\centering

\begin{subfigure}{0.5\textwidth}
\includegraphics[width=\linewidth]{{submodule/dishtiny/binder/bucket=prq49/a=adaptation_assays+endeavor=16/teeplots/assay-subject=Specimen+col=contemporary-biotic-background+kind=count+row=prefatory-biotic-background+viz=barlabel-catplot+x=without-biotic-background+ext=}}
\caption{Joint distribution of adaptation assay on representative specimen from focal strain over biotic backgrounds, with diversity maintenance during competition.}
\label{fig:outcome_count_joint_distns:specimen_with_dm}
\end{subfigure}%
\begin{subfigure}{0.5\textwidth}
\includegraphics[width=\linewidth]{{submodule/dishtiny/binder/bucket=prq49/a=adaptation_assays+endeavor=16/teeplots/assay-subject=Specimen+col=contemporary-no-diversity-maint-biotic-background+kind=count+row=prefatory-no-diversity-maint-biotic-background+viz=barlabel-catplot+x=without-biotic-background+ext=}}
\caption{Joint distribution of adaptation assay on representative specimen from focal strain over biotic backgrounds, with diversity maintenance disabled during competition.}
\label{fig:outcome_count_joint_distns:specimen_no_dm}
\end{subfigure}

\begin{subfigure}{0.5\textwidth}
\includegraphics[width=\linewidth]{{submodule/dishtiny/binder/bucket=prq49/a=adaptation_assays+endeavor=16/teeplots/assay-subject=Population+col=contemporary-biotic-background+kind=count+row=prefatory-biotic-background+viz=barlabel-catplot+x=without-biotic-background+ext=}}
\caption{Joint distribution of adaptation assay on focal strain population over biotic backgrounds, with diversity maintenance during competition.}
\label{fig:outcome_count_joint_distns:population}
\end{subfigure}


\caption{
Joint distribution of adaptation assay outcomes across biotic backgrounds.
For each adaptation assay, three outcomes were possible: significant fitness gain, significant fitness loss, or no significant fitness change (``neutral'').
Significance cutoff $p < 0.005$ was used.
A fitness loss --- color coded red --- corresponds to winning 2 or fewer competitions out of 20 against the preceding stint's focal strain population.
A fitness gain --- color coded green --- corresponds to winning 18 or more competitions out of 20 against the preceding stint's focal strain population.
Neutral fitness outcomes are color coded yellow.
Outcome counts are accumulated over experiments from stint 1 through stint 100.
Counts in each subfigure therefore sum to 100.
Column position in facet grid indicates outcome with contemporary biotic background, row position indicates outcome with prefatory biotic background, and bar color and $x$ position indicates outcome without biotic background.
See Figure \ref{fig:adaptation_assay_cartoon} for explanation of competition biotic backgrounds.
See Figure \ref{fig:with_vs_without_diversity_maintenance} for detail on joint distribution of outcomes with and without diversity maintenance, which were mostly identical.
}
\label{fig:outcome_count_joint_distns}
\end{figure*}


After incorporating the background strain into our measure of fitness, we detected significantly fewer whole-population deleterious outcomes --- six under contemporary biotic background conditions and three under prefatory biotic background conditions ($p = 0.04$ and $p = 0.028$, Fisher's exact two-tailed test; Figure \ref{fig:outcome_count_distns}).
It is important to note, however, that this result may be influenced by slower simulation execution under biotic background conditions (Supplementary Figure \ref{fig:num_updates_elapsed_barplot}).

However, we did find stronger direct evidence of a selective effect by the background strain: four whole-population outcomes that were deleterious without biotic background were actually significantly advantageous in the presence of both the prefatory and contemporary background strain populations (Figure \ref{fig:outcome_count_joint_distns:population}).
Additionally, two deleterious outcomes without biotic background were detected as significantly adaptive under the prefatory biotic background but were neutral under contemporary biotic background (Figure \ref{fig:outcome_count_joint_distns:population}).
A further five outcomes were detected as significantly deleterious without biotic background but were not detected as significantly adaptive or deleterious (i.e., neutral) in the presence  of both background strain populations (Figure \ref{fig:outcome_count_joint_distns:population}).
We also found one whole-population outcome that was significantly advantageous without biotic background and in the presence of the prefatory background strain population but significantly deleterious in the presence of the contemporary background strain, possibly suggesting a ``arms race''-esque evolutionary innovation on the part of the background strain over that stint (Figure \ref{fig:outcome_count_joint_distns:population}).

We still saw three whole-population outcomes that were significantly deleterious under all three conditions (Figure \ref{fig:outcome_count_joint_distns:population}).
Muller's ratchet \citep{andersson1996muller} or maladaptation due to environmental change \citep{brady2019causes} may provide possible explanations, but a definitive answer will require further study.

We also performed fitness assays on individual sampled specimens with both biotic backgrounds.
Out of 100 stints tested, we observed 20 significantly deleterious outcomes without biotic background, 23 significantly deleterious outcomes under prefatory biotic background, and 12 significantly deleterious outcomes under contemporary biotic background (Figure \ref{fig:outcome_count_distns:specimen_with_dm}).
Reciprocally, we observed 57 significantly adaptive outcomes without biotic background, 44 with prefatory biotic background, and 48 with contemporary biotic background (Figure \ref{fig:outcome_count_distns:specimen_with_dm}).
Greater sensitivity of the ``without biotic background'' adaptation assay could account for the counterintuitive detection of more adaptive outcomes under foreign environmental biotic conditions (i.e., the absence of the background strain).

As before with the population-level adaptation assays, we detected four specimen outcomes that were deleterious without biotic background but significantly advantageous under both tested background strain populations (Figure \ref{fig:outcome_count_joint_distns:specimen_with_dm}).
Additionally, and again as before, two deleterious outcomes without biotic background were detected as significantly adaptive under the prefatory biotic background but neutral under the contemporary biotic background (Figure \ref{fig:outcome_count_joint_distns:specimen_with_dm}).

We found no specimen outcomes that were advantageous under the prefatory biotic background but deleterious under the contemporary background.
However, we found three stints with opposite dynamics: specimen outcomes deleterious under prefatory biotic background but advantageous under contemporary biotic background (Figure \ref{fig:outcome_count_joint_distns:specimen_with_dm}), further suggesting coincident, interacting evolutionary innovations along focal and background strain lineages (Figure \ref{fig:outcome_count_joint_distns:specimen_with_dm}).

\begin{figure*}
\centering

\begin{subfigure}{0.5\textwidth}

\includegraphics[width=\linewidth]{{submodule/dishtiny/binder/bucket=prq49/a=adaptation_assays+endeavor=16/teeplots/assay-subject=Specimen+hue=prefatory-no-diversity-maint-biotic-background+kind=count+viz=barlabel-catplot+x=prefatory-biotic-background+ext=}}
\caption{TODO}
\end{subfigure}%
\begin{subfigure}{0.5\textwidth}
\includegraphics[width=\linewidth]{{submodule/dishtiny/binder/bucket=prq49/a=adaptation_assays+endeavor=16/teeplots/assay-subject=Specimen+hue=contemporary-no-diversity-maint-biotic-background+kind=count+viz=barlabel-catplot+x=contemporary-biotic-background+ext=}}
\caption{TODO}
\end{subfigure}

\caption{
TODO
}
\label{fig:with_vs_without_diversity_maintenance}
\end{figure*}


To better characterize the mechanism of fitness effects caused by the background strain, we performed additional competition experiments with sampled specimens in the presence of the background strain population with diversity maintenance disabled.
This allowed us to test whether action of the diversity maintenance mechanism, rather than direct interactions between the focal and background strains, caused the observed fitness effects.
Figure \ref{fig:with_vs_without_diversity_maintenance} compares adaptation assay outcomes with and without diversity maintenance under both the prefatory and contemporary biotic background conditions.
Outcomes were generally similar, and only one sign-change difference was observed: one specimen outcome was beneficial under prefatory biotic background conditions without diversity maintenance but deleterious with diversity maintenance.
Further, as shown in Figure \ref{fig:outcome_count_joint_distns:specimen_without_dm}, we observed near-identical sign-change fitness effects of biotic background as noted above.
So, biotic selective effects cannot be explained as an artifact of activation of the diversity maintenance scheme.

\begin{sidewaysfigure*}
\centering
\includegraphics[width=\linewidth]{{submodule/dishtiny/binder/bucket=prq49/a=adaptation_assays+endeavor=16/teeplots/hue=fitness-gain-or-loss+viz=facet-heatmap+x=biotic-background+y=competition-stint+ext=}}

\caption{
Summary of adaptation assay outcomes for sampled representative specimen (top) population-level adaptation (bottom).
}
\label{fig:fitness_gain_or_loss}
\end{sidewaysfigure*}


Significant increases in fitness occur throughout the evolutionary history of the case study, but not at every stint.
Figure \ref{fig:fitness_gain_or_loss.tex} summarizes the outcome of all adaptation assays stint-by-stint across evolutionary history.
Neutral outcomes appear to occur more frequently at later stints.
This may be indicative of slower evolutionary innovation, but may also result to some extent from simulation of fewer generations during evolutionary stints (Supplementary Figure \ref{fig:simulation}) and during competition experiments (Supplementary Figure \ref{fig:num_updates_elapsed_barplot}) due to slower execution of later genomes.

\begin{sidewaysfigure*}
\centering
\includegraphics[width=\linewidth]{{submodule/dishtiny/binder/bucket=prq49/a=adaptation_assays+endeavor=16/teeplots/hue=symlog-median-fitness-differential+viz=facet-heatmap+x=biotic-background+y=competition-stint+ext=}}

\caption{
Median calculated fitness differential outcomes of competition experiments.
Zero fitness differential corresponds to a neutral result, color mapped to white.
Blue indicates positive fitness differential (fitness gain) compared to the previous stint and red indicates negative fitness differential (fitness loss).
Color coding and parentheticals of stint labels correspond to qualitative morph codes described in Table \ref{tab:morph_descriptions}.
Note that color intensity is plotted on a symlog scale due to distribution of fitness differentials over multiple orders of magnitude.
Upper panels shows results for sampled focal strain genome, lower panel shows results for entire focal strain population.
See Figure \ref{fig:adaptation_assay_cartoon} for explanation of competition biotic backgrounds.
}
\label{fig:median_fitness_differential_symlog}
\end{sidewaysfigure*}


Figure \ref{fig:median_fitness_differential_symlog} shows the magnitudes of calculated fitness differentials for all adaptation assays.
Fitness differentials during the first 40 stints are generally higher magnitude than later fitness differentials, although a very strong fitness differential occurs at stint 93.
Although the emergence of morphology $d$ was associated with significant increases in fitness in some specimen assays and morphologies $e$ and $g$ were associated with significant increases in fitness across all specimen assays (Figure \ref{fig:fitness_gain_or_loss}), the magnitude of these fitness differentials appears ordinary compared to fitness differentials at other stints (Figure \ref{fig:fitness:fitness_magnitude}).
Supplementary Figure \ref{fig:mean_competition_prevalence} shows mean end-competition prevalence across assays, telling a similar story.

\begin{subfigure}{0.5\textwidth}

\includegraphics[width=\linewidth]{{plots/fitness/bucket=prq49+cat=morph+endeavor=16+transform=filter-Series-16005+viz=letterscatter-vline+x=stint+y=mean-doubling-time-growth-rate+ext=}}

\end{subfigure}%

In addition to competition assays, we also measured growth rate of specimen strains by tracking doubling time (in updates) when seeded into quarter-full toroidal grids (Figure \ref{fig:doubling_time}).
Morph $b$ exhibited a fast growth rate early on that was never matched by later morphs.
This measure appears to be a poor overall proxy for fitness, highlighting the importance of biotic aspects of the simulation environment (which are not present in the empty space the assayed cells double into).


\subsection{Fitness Complexity}

\begin{figure}

\begin{subfigure}{0.5\textwidth}

\includegraphics[width=\linewidth]{{plots/critical_fitness_complexity/bucket=prq49+cat=morph+endeavor=16+transform=filter-Series-16005+viz=letterscatter-vline+x=stint+y=critical-fitness-complexity+ext=}}

\caption{
Critical fitness complexity.
Number of single-site nopouts that significantly decrease fitness, adjusted for expected false positives.
}
\label{fig:fitness_complexity:critical_fitness_complexity}

\end{subfigure}%

\begin{subfigure}{0.5\textwidth}

\includegraphics[width=\linewidth]{{plots/composite_fitness_complexity/bucket=prq49+cat=morph+ci=95+endeavor=16+transform=filter-Series-16005+viz=letterscatter-err-vline+x=stint+y=composite-fitness-complexity+ext=}}

\caption{
Composite fitness complexity.
Sum of critical fitness complexity and interpolated fitness complexity.
Error bars represent 95\% credible interval.
Vertical gray bars represent missing estimates due to bad phenotype-neutral nopouts.
Supplementary Figure \ref{fig:fitness_complexity_alt:composite_fitness_complexity_alt} shows the same data with smaller markers.
}
\label{fig:fitness_complexity:composite_fitness_complexity}

\end{subfigure}%
\begin{subfigure}{0.5\textwidth}

\includegraphics[width=\linewidth]{{plots/interpolated_fitness_complexity/bucket=prq49+cat=morph+endeavor=16+transform=filter-Series-16005+viz=letterscatter-err-vline+x=stint+y=interpolated-fitness-complexity+ext=.pdf}}

\caption{Interpolated fitness complexity.
Estimated number of non-critical sites that contribute to fitness.
Vertical gray lines represent stints with missing estimates due to bad phenotype-neutral nopouts.
Shaded areas represent 95\% confidence intervals.
Supplementary Figure \ref{fig:fitness_complexity_alt:interpolated_fitness_complexity_alt} shows the same data with smaller markers.
}
\label{fig:fitness_complexity:interpolated_fitness_complexity}

\end{subfigure}%

\caption{ TODO fitness complexity }
\label{fig:fitness_complexity}

\end{figure}

Figure \ref{fig:fitness_complexity:critical_fitness_complexity} plots critical fitness complexity of specimens drawn from across the case study's evolutionary history.

Critical fitness complexity reaches more than 20 under morph $b$, jumps to more than 40 under morph $d$, drops to slightly more than 30 for morph $e$.
Critical fitness complexity reaches a peak of 48 sites around stint 39 then levels out and decreases.
This decrease may in part be due to declining sensitivity of competition experiments due to slower simulation resulting in execution of fewer updates within the fixed-duration jobs (Supplementary Figure \ref{fig:num_updates_elapsed}).

Phylogenetic analysis (Figure \ref{fig:phylo_parsimony_tree}) suggests independent origins of the critical fitness complexity in morph $d$ and morph $e$ --- the morph $d$ specimen from stint 14 is more closely related to the morph $b$ specimen from stint 13 than to the morph $e$ specimen from stint 15.
Likewise, specimens of lower complexity morphs $i$ and $b$ that appear past stint 70 appear to have independent evolutionary origins.

% The evolution of morph $g$ is not associated with a clear change in critical fitness complexity.

% Interpolated fitness complexity remains much more steady over the course of evolutionary history, although the certainty of our estimate of interpolated fitness complexity varies greatly.
% Figure \ref{fig:fitness_complexity:interpolated_fitness_complexity} shows this.
% Sometimes we were not able to estimate composite fitness complexity because the phenotype neutral nopout was not truly neutral --- it was significantly less fit than wildtype.

% Figure \ref{fig:fitness_complexity:composite_fitness_complexity} shows composite fitness complexity, the sum of interpolated fitness complexity and critical fitness complexity.
% The highest best estimate of composite fitness complexity was 64 sites at stint 61.
% The highest lower bound estimate of composite fitness complexity was 49 sites at stint 70.

\subsection{Interface Complexity}

\input{lib/dissertationonly.tex}
\begin{figure*}
\dissertationonly{\captionsetup[subfigure]{font=scriptsize}}
\dissertationonly{\captionsetup{font=footnotesize}}
\begin{subfigure}{0.5\textwidth}

\includegraphics[width=\linewidth]{{plots/cardinal_interface_complexity/bucket=prq49+cat=morph+endeavor=16+transform=filter-Series-16005+viz=letterscatter+x=stint+y=cardinal-interface-complexity+ext=}}

\caption{Cardinal interface complexity, the total number of distinct interactions between a virtual CPU controlling cell behavior and its surroundings that contribute to fitness.
(Sum of Figures \ref{fig:interface_complexity:extrospective_interface_complexity}, \ref{fig:interface_complexity:introspective_interface_complexity}, \ref{fig:interface_complexity:writable_interface_complexity}, \ref{fig:interface_complexity:intermessage_interface_complexity}, and \ref{fig:interface_complexity:intramessage_interface_complexity}.)}
\label{fig:interface_complexity:cardinal_interface_complexity}

\end{subfigure}%
\begin{subfigure}[t]{0.49\textwidth}

\includegraphics[width=\linewidth]{{plots/intermessage_interface_complexity/bucket=prq49+cat=morph+endeavor=16+transform=filter-Series-16005+viz=letterscatter-vline+x=stint+y=num-less-fit-under-inter-self-send-filter-mod-20+ext=}}

\caption{Intermessage interface complexity, the number of distinct inter-cell messages that contribute to fitness.}
\label{fig:interface_complexity:intermessage_interface_complexity}

\end{subfigure}%

\begin{subfigure}{0.5\textwidth}

\includegraphics[width=\linewidth]{{plots/intramessage_interface_complexity/bucket=prq49+cat=morph+endeavor=16+transform=filter-Series-16005+viz=letterscatter-vline+x=stint+y=num-less-fit-under-intra-self-send-filter-mod-20+ext=}}

\caption{Intramessage interface complexity, the number of distinct inter-cell messages that contribute to fitness.}
\label{fig:interface_complexity:intramessage_interface_complexity}

\end{subfigure}%
\begin{subfigure}{0.5\textwidth}

\includegraphics[width=\linewidth]{{plots/introspective_interface_complexity/bucket=prq49+cat=morph+endeavor=16+transform=filter-Series-16005+viz=letterscatter-vline+x=stint+y=num-less-fit-under-introspective-state-perturbation+ext=}}

\caption{Introspective interface complexity, the number of states viewed in the own cell that contribute to fitness. See Supplementary Figure \ref{fig:introspective_perturbation} for detail on the introspective states that contribute to fitness.}
\label{fig:interface_complexity:introspective_interface_complexity}

\end{subfigure}%

\begin{subfigure}{0.5\textwidth}

\includegraphics[width=\linewidth]{{plots/extrospective_interface_complexity/bucket=prq49+cat=morph+endeavor=16+transform=filter-Series-16005+viz=letterscatter-vline+x=stint+y=num-less-fit-under-extrospective-state-perturbation+ext=}}

\caption{ Extrospective interface complexity, the number of states viewed in neighboring cells that contribute to fitness.
See Supplementary Figure \ref{fig:extrospective_perturbation} for detail on the extrospective states that contribute to fitness.
}
\label{fig:interface_complexity:extrospective_interface_complexity}

\end{subfigure}%

\begin{subfigure}{0.5\textwidth}

\includegraphics[width=\linewidth]{{plots/writable_interface_complexity/bucket=prq49+cat=morph+endeavor=16+transform=filter-Series-16005+viz=letterscatter-vline+x=stint+y=num-less-fit-under-writable-state-perturbation+ext=}}

\caption{Writable state interface complexity, the number of output states that contribute to fitness. See Supplementary Figure \ref{fig:writable_perturbation} for detail on the writable states that contribute to fitness.}
\label{fig:interface_complexity:writable_interface_complexity}

\end{subfigure}%


\caption{ Interface complexity estimates. Color coding and letters correspond to qualitative morph codes described in Table \ref{tab:morph_descriptions}.
Dotted vertical line denotes emergence of morph $e$.
Dashed vertical line denotes emergence of morph $g$.}
\label{fig:interface_complexity}
\end{figure*}


Figure \ref{fig:interface_complexity} summarizes cardinal interface complexity, as well as its constituent components, for specimens drawn from across the case study's evolutionary history.

%todo: not a sentence
Notably, cardinal interface complexity more than doubles from 6 interactions to 17 interactions coincident with the emergence of morph $e$ (Figure \ref{fig:interface_complexity:cardinal_interface_complexity}).
This is due to simultaneous increases in extrospective state sensing (2 to 9 states; Figure \ref{fig:interface_complexity:extrospective_interface_complexity}), introspective state sensing (1 to 4 states; Figure \ref{fig:interface_complexity:introspective_interface_complexity}), and writable state usage (1 to 2 states; Figure \ref{fig:interface_complexity:writable_interface_complexity}).

The emergence of morph $g$ coincided with an increase in writable state interface complexity from 1 to 3 as shown in Figure \ref{fig:interface_complexity:writable_interface_complexity}.
However, morph $g$ was not associated with other changes in other aspects of cardinal interface complexity.
The greatest observed cardinal interface complexity was 22 interactions at stints 54 and 67.
