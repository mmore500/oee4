% \pragmaonce

% adapted from https://www.overleaf.com/learn/latex/Commands
\providecommand{\dissertationexclude}[1]{%
% adapted from https://tex.stackexchange.com/a/33577
\ifdefined\DISSERTATION
\else
#1
\fi
}


\section{Introduction}

The challenge, and promise, of open-ended evolution has animated decades of inquiry and discussion within the artificial life community \citep{packard2019overview}.
The difficulty of devising models that produce continuing open-ended evolution suggests profound philosophical or scientific blind spots in our understanding of the natural processes that gave rise to contemporary organisms and ecosystems.
Already, pursuit of open-ended evolution has yielded paradigm-shifting insights.
For example, novelty search demonstrated how processes promoting non-adaptive diversification can ultimately yield adaptive outcomes that were previously unattainable \citep{lehman2011abandoning}.
Such work lends insight to fundamental questions in evolutionary biology, such as the relevance --- or irrelevance --- of natural selection with respect to increases in complexity \citep{lehman2012evolution, Lynch8597} and the origins of evolvability \citep{lehman2013evolvability,Kirschner8420}.
Evolutionary algorithms devised in support of open-ended evolution models also promise to deliver tangible broader impacts for society.
Possibilities include the generative design of engineering solutions, consumer products, art, video games, and AI systems \citep{nguyen2015,stanley2019open}.

Preceding decades have witnessed advances toward defining --- quantitatively and philosophically --- the concept of open-ended evolution \citep{lehman2012beyond,dolson2019modes,bedau1998classification} as well as investigating causal phenomena that promote open-ended dynamics such as ecological dynamics, selection, and evolvability \citep{dolson2019constructive,soros2014identifying,huizinga2018emergence}.
The concept of open-endedness is fundamentally characterized by intertwined generation of novelty, functional complexity, and adaptation \citep{taylor2016open}.
How and how closely these phenomena relate to one another remains an open question.
Here, we aim to complement ongoing work to develop a firmer theoretical understanding of the relationship between novelty, complexity, and adaptation by exploring the evolution of these phenomena through a case study using the DISHTINY digital multicelullarity framework \dissertationexclude{\citep{moreno2019toward}}.
We apply a suite of qualitative and quantitative measures to assess how these qualities can change over evolutionary time and in relation to one another.
