\section{Conclusion}

Complexity and novelty are not inevitable outcomes of Darwinian evolution \citep{stanley2017open}.
Instead, how and why some lineages within some model systems evolve complexity and novelty merits explanation.
In this regard, efforts to develop substrates and conditions sufficient to observe meaningful evolution of complexity and novelty play a crucial role in assessing the sufficiency of theory.
% Additionally, subsequent availability of complexity and novelty potential within experimental substrates enables work to test and refine theory.
The artificial life research community has a rich track record in such work.

The case study reported here tracks a lineage over two phenotypic innovations and several-fold increases in complexity.
DISHTINY relaxes common simulation constraints \citep{goldsby2012task, goldsby2014evolutionary}, enabling broad genetic determination of multicellular life history and allowing for unconstrained cellular interactions between multicellular bodies.
As such, this case study opens a concrete window into evolutionary origins of complexity and novelty under regimes of strong biotic interactions.

Our case study suggests a loose coupling between novelty, complexity, and adaptation.
We observe instances where novelty coincides with adaptation and instances where it does not.
We observe instances where increases in complexity coincide with adaptation and where decreases in complexity coincide with adaptation.
We observe instances where innovation coincides with spikes in complexity and instances where it does not.
We even observe substantial divergences between metrics measuring different aspects of functional complexity.
For example, the specimen sampled at stint 15 had nearly triple the interface complexity of the specimen sampled at stint 14 but lower critical fitness complexity.

Loose coupling between the conceptual threads of novelty, complexity, and adaptation in this case study highlights the importance of considering these factors independently when developing theory around open-ended evolution --- inherent coupling cannot be assumed.

Our observation of significant selective effects by the background strain suggests that it may serve a crucial role in understanding the focal strain.
Future work should characterize trajectories of adaptation, novelty, and complexity in this background strain.
Additionally, success of the biotic background in fleshing out our adaptation assays suggests that complexity measures could be improved through similar incorporation of the biotic background.
It would be particularly interesting to measure the contribution of the background strain to complexity as the difference between complexity statistics with and without the biotic background.
To more systematically test the role of biotic selection on facilitating evolution of complexity, future experiments might test for differences in the rate of high-complexity evolutionary outcomes between evolution experiments with and without long-term coexistence between lineages (i.e., with diversity maintenance mechanisms enabled versus disabled).

This case study highlights the potential usefulness of toolbox-based approaches to analyzing open-ended evolution systems in which an array of analyses are performed to distinguish disparate dimensions of open-endedness \citep{dolson2019modes}.
Our findings underscore, in particular, the critical role of biotic context in such analyses.
In future work, we are interested in further extending this toolbox.
One priority will be estimating epistatic contributions to fitness without resorting to all-pairs knockouts or other even more extensive assays \citep{moreno2024methods}.
Such methodology will be crucial for systems where fitness is implicit and expensive to measure.


% In future work, we are interested in developing statistical methodology to estimate without .


% This
% Future work exploring open-endedness should continue along the lines of the MODES framework , building a broad analytical toolbox that can distinguish disparate dimensions of open-endedness.

% Continuing the work of MODES to broaden a toolbox rather than focus in on a single metric.

% Serves as a trial balloon for future, more systematic exploration of the relationship between novelty, complexity, and adaptation.

% For example, if that only activated in response to the other lineage then couldn't detect that

% Especially in systems where fitness is implicit it's costly or impractical to do comprehensive analyses.
% We need to continue develop methodology to statisticaly sample

% Problem also that fitness depends on conditions, so there may be problems replicating conditions of evolution environment (presence of mutation, presence of other strains).
% Real-time proof-of-concept but requires more thought about methodology to control for the slowing of evolutionary pace.


% We use the idea of in addition to existing ideas about sequence complexity.
% Interface complexity the richness of interactions between an agent and its environment.
% 8 After all, the central concept of adaptation—``the design or suitability of an object for a particular
% function'' (Gould 2002, p. 117)—directly refers to functionality. A few examples of those taking
% biological functionality as integral to biological complexity include Bronowski (1970), Wicken (1979),
% Dawkins (1986), Kampis and Csa´nyi (1987), Carroll (2001), Adami (2002) \citep{korb2011evolution}

% The question of the relation between sequence complexity and functional complexity is.
% Sequence complexity is defined in terms of entropy of , but in a practical sense it can be approximated by looking at critical fitness complexity.

% a novel innovation is accompanied by a decrease in sequence/critical fitness complexity but an increase in interface complexity.

% Already known in Avida, where sequence/critical fitness complexity would decrease after a innovation as became more adapted

% Surprisingly, the novelty wasn't accompanied by out-of-the ordinary fitness differential.

% However, we present a case study where

% If $d$ is a member of another lineage, then composite fitness complexity didn't increase greatly even though interface complexity more than doubled.

% A second novelty occurs without a clear signal with respect to sequence/fitness complexity or interface complexity.

% Methodology to quantify is particularly critical.
% One area of interest is understanding the functional complexity of .
% Phenotype complexity, Interface complexity and statistical methodololgy to sample stuff.
% We advance proposed methodology by dolson et al with respect to identifying

% We look at a case study to see how these complexity metrics change over an evolutionary history with qualitative moprhological innovations.
