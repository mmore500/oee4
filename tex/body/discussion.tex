\section{Discussion}

Throughout the case study lineage, we describe an evolutionary sequence of ten qualitatively distinct multicellular morphologies (Table \ref{tab:morph_descriptions}).
The emergence of some, but not all, of these morphologies coincided with an increase in fitness compared to the preceding population.
Outcomes from the first observed morphology $c$ specimen are significantly deleterious in all contexts.
Likewise, morphology $f$, while advantageous in the absence of the background strain, appeared neutral in its presence (Figure \ref{fig:fitness_gain_or_loss}).
However, the genesis of morphology $e$, and $g$ are associated with significant fitness gain in all contexts (Figure \ref{fig:fitness_gain_or_loss}).
This latter set of novelties might be described as ``innovations,'' which Hochberg et al. define as qualitative novelty associated with an increase in fitness \citep{hochberg2017innovation}.
Interestingly, the magnitude of the fitness differentials associated with the emergence of morphologies $e$, and $g$ do not appear to fall outside the bounds of other stint-to-stint fitness differentials (Figure \ref{fig:median_fitness_differential_symlog}).

The relationship between innovation and complexity also appears to be loosely coupled.
The emergence of morphology $d$ was accompanied by a spike in critical fitness complexity (from 25 sites at stint 13 to 43 sites at stint 14).
However, the emergence of morphology $i$ coincided with a loss of critical fitness complexity (from more than 30 sites to fewer than 10 sites).
The specimen of morph $i$ at stint 77, which phylogenetic analysis suggests may have independent trait origin from the specimen at stint 75, exhibited significant fitness gain across all contexts despite decimation of complexity.

Phylogenetic analysis suggests that morphology $e$ was not a direct descendant of morphology $d$.
So the emergence of morphology $e$ appears to have coincided with a more modest increase in fitness complexity from 25 sites to 31 sites.
Similarly, the emergence of morphology $g$ with 42 critical sites at stint 45 coincided with a relatively modest increase in fitness complexity from 39 critical sites at stint 44.

We also see evidence that increases in complexity do not imply qualitative novelty in morphology.
In Figure \ref{fig:critical_fitness_complexity}, we can also observe notable increases in critical fitness complexity that did not coincide with apparent morphological innovation.
For example, fitness complexity jumped from 11 sites at stint 11 to 27 sites at stint 12 while morphology $b$ was retained.
In addition, a more gradual increase in fitness complexity was observed from 27 sites at stint 16 to 46 sites at stint 36 all with consistent morphology $e$.

Finally, we also observed disjointedness between alternate measures of functional complexity.
Notably, critical fitness complexity increased by 18 sites with the emergence of morph $d$ but interface complexity increased only marginally.
Subsequently observed morph $e$ had nearly triple the interface complexity of morph $d$ (6 interactions vs. 17 interactions) but had 12 sites lower critical fitness complexity.
In addition, the gradual increase in critical fitness complexity between stint 15 and 36 under morphology $e$ is not accompanied by a clear change in interface complexity (Figures \ref{fig:interface_complexity:cardinal_interface_complexity} and \ref{fig:fitness_complexity:critical_fitness_complexity}).
These apparent inconsistencies between metrics for functional complexity evidence the multidimensionality of this idea and underscore well-known difficulties in attempts to describe and quantify it \citep{bottcher2018molecules}.
