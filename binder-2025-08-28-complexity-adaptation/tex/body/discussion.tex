\section{Discussion}

Across the case study lineage, we describe an evolutionary sequence of ten qualitatively distinct multicellular morphologies (Table \ref{tab:morph_descriptions}).
The emergence of some, but not all, of these morphologies coincided with an increase in fitness compared to the preceding population.
Fitness assays from the sole observed specimen of morphology $c$ are significantly deleterious in all contexts.
Similarly, the sole observed specimen of morphology $f$ exhibited an advantage only in the absence of the background strain (Figure \ref{fig:fitness_gain_or_loss}).
By contrast, the geneses of morphologies $e$ and $g$ were associated with significant fitness gain in all contexts (Figure \ref{fig:fitness_gain_or_loss}).
Likewise, the sole observed specimen of morphology $d$ exhibited a significant fitness advantage in most, but not all, tested contexts.
This latter set of novelties ($e$, $g$, and $d$) might be described as ``innovations,'' which \citet{hochberg2017innovation} define as qualitative novelty associated with an increase in fitness.
Interestingly, though, the fitness effect sizes observed from these ``innovations'' appear generally comparable in magnitude to other stint-to-stint trials with no substantial novelties (Figure \ref{fig:median_fitness_differential_symlog}).

The relationship between innovation and complexity appears to be loosely coupled.
On one hand, the origination of morphologies $d$ and $e$, respectively, coincided with a critical fitness complexity spike (25 to 43 sites) and an interface complexity spike (5 to 17 interactions).%
\footnote{Note that phylogenetic analysis suggests that morphology $e$ was not a direct descendant of morphology $d$, but that both instead descended from morphology $b$.}
The emergence of morphology $g$, by contrast, coincided with more modest changes in critical fitness complexity and interface complexity (41 to 42 sites and 12 to 15 interactions).
At the opposite extreme, consider the morph $i$ origination observed at stint 75, which exhibited a significant fitness advantage across all tested contexts.
In this case, innovation coincided with sharp decreases in both critical fitness complexity (33 to 3 sites) and interface complexity (12 to 0 interactions).

Although increases in complexity accompanied morphological novelty in several cases, we also observed substantial changes in complexity occurring outside this context.
% In Figure \ref{fig:critical_fitness_complexity}, we can also observe notable increases in critical fitness complexity that did not coincide with apparent morphological innovation.
For example, a spike in critical fitness complexity (11 to 27 sites) was observed between stints 11 and 12 --- with morphology $b$ exhibited at both timepoints.
Similarly, morphology $e$ was retained across a long, gradual increase in critical fitness complexity (31 to 46 sites) between stints 15 and 36.

Finally, we also observed notable disjointedness between alternate measures of functional complexity.
For instance, critical fitness complexity increased by 18 sites with the emergence of morph $d$, but interface complexity increased only marginally (5 to 6 interactions).
Further, compared against morph $d$, the first specimen of morph $e$ exhibited near triple the interface complexity (6 vs. 17 interactions), but 12 fewer sites of critical fitness complexity.
Consider also the increase in critical fitness complexity between stints 15 and 36 under morphology $e$ (31 to 46 sites), which is not accompanied by a clear change in interface complexity (Figures \ref{fig:interface_complexity:cardinal_interface_complexity} and \ref{fig:critical_fitness_complexity}).
These discrepancies between alternate metrics for functional complexity suggest underlying multidimensionality, and underscore well-known difficulties in attempts to describe and quantify complexity \citep{bottcher2018molecules}.
