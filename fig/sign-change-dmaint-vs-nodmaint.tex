\begin{figure*}

\includegraphics[width=0.51\linewidth,trim={0 0 0 0cm},clip]{%
binder-2025-08-24-keyfig/binder/teeplots/2025-08-24-keyfig/baseline=contemporary+bbg=contemporary-no-diversity-maint+subject=specimen+viz=signchangedotplot+ext=.pdf}%
\includegraphics[width=0.45\linewidth,trim={1.2cm 0 0 0cm},clip]{%
binder-2025-08-24-keyfig/binder/teeplots/2025-08-24-keyfig/baseline=prefatory+bbg=prefatory-no-diversity-maint+subject=specimen+viz=signchangedotplot+ext=.pdf}

\vspace{-1ex}

\caption{
\textbf{Screen for fitness trade-offs driven by diversity maintenance scheme.}
\footnotesize
Plots report influence on focal strain competition assays attributable to diversity maintenance mechanism.
Experiments competed focal strain samples (specimen from stint $n+1$ against baseline focal strain population from stint $n$.
Outcomes recorded between $1$ (fitness gain; stint $n$ baseline driven extinct) and $-1$ (fitness loss; stint $n+1$ driven extinct), with near-zero values indicating comparable fitness.
Vertical line segments mark discrepancy between conditions with diversity maintenance (hollow markers) and without diversity maintenance (filled markers).
Markers omitted where no significant significant fitness change detected between stint $n+1$ and stint $n$ baseline ($\alpha = 0.005$).
Segment color highlights sign-change effect: harmful trait becomes beneficial without diversity maintenance mechanism (blue) or vice versa (red).
Subplots differ in source of ecological context (columns).
Under prefatory context, background strain from stint $n$; under contemporary context, stint $n+1$.
}
\label{fig:sign-change-nodmaint-vs-nodmaint}

\end{figure*}
