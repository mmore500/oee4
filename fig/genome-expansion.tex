\begin{figure}

\includegraphics[width=0.39\linewidth]{%
binder-2025-09-05-genome-expansion-fitness/binder/teeplots/2025-09-05-genome-expansion-fitness/biotic_background=Contemporary+hue=genome-expansion+palette=pastel1+subject=Specimen+test=mw+viz=violinplot+x=genome-expansion+y=fitness-differential-focal+ext=.pdf}
\includegraphics[width=0.305\linewidth, trim={1.3cm 0 0 0.6cm}, clip]{%
binder-2025-09-05-genome-expansion-fitness/binder/teeplots/2025-09-05-genome-expansion-fitness/biotic_background=Prefatory+hue=genome-expansion+palette=pastel1+subject=Specimen+test=mw+viz=violinplot+x=genome-expansion+y=fitness-differential-focal+ext=.pdf}%
\includegraphics[width=0.305\linewidth, trim={1.3cm 0 0 0.6cm}, clip]{%
binder-2025-09-05-genome-expansion-fitness/binder/teeplots/2025-09-05-genome-expansion-fitness/biotic_background=Without+hue=genome-expansion+palette=pastel1+subject=Specimen+test=mw+viz=violinplot+x=genome-expansion+y=fitness-differential-focal+ext=.pdf}

\vspace{-1ex}

\begin{subfigure}{0.135\linewidth}
~
\end{subfigure}%
\begin{subfigure}{0.305\linewidth}
    \centering
    \caption{\footnotesize contemporary\\background}
    \label{fig:genome-expansion:contemporary}
\end{subfigure}%
\begin{subfigure}{0.305\linewidth}
    \centering
    \caption{\footnotesize prefatory\\background}
    \label{fig:genome-expansion:prefatory}
\end{subfigure}%
\begin{subfigure}{0.255\linewidth}
    \centering
    \caption{\footnotesize without\\background}
    \label{fig:genome-expansion:without}
\end{subfigure}

\caption{
    \textbf{Association of adaptation with genome expansion.}
    \footnotesize
    Violin plots compare fitness differentials from specimens with genome size increase relative to ancestor, versus those without.
    Fitness assays compete focal specimen at stint $n+1$ against focal population at stint $n$.
    When competed in the presence of background strain population from stint $n+1$, genome expansions accompany stronger adaptation --- with small effect size (panel \ref{fig:genome-expansion:contemporary}; $p = 0.013, \; \delta = 0.28$).
    When competed with no biotic background, a similar trend appears under fitness assays without biotic background, although not significant under a two-tailed alternative hypothesis (panel \ref{fig:genome-expansion:without}; $p=0.08, \; \delta = 0.22$).
    By contrast, no effect is apparent with prefatory biotic background from stint $n$ (panel \ref{fig:genome-expansion:prefatory}).
    Reported statistics are Cliff's delta (effect size) and two-tailed Mann-Whitney U test (significance).
    Two-sample $t$-tests (two-tailed alternative) gave similar results.
    Independent adaptation assays, conducted on focal specimens without balancing procedures for diversity maintenance with background strains, gave corroborating results (Supplementary Figure \ref{fig:genome-expansion-nodmaint}).
    Contemporary background assays exhibited significant adaptation-expansion association ($p=0.02, \; \delta=0.27$), but prefatory background assays did not ($p=0.32, \; \delta=0.12$).
    Further yet, consistent patterns arose in adaptation assays on whole focal-strain populations (Supplementary Figure \ref{fig:genome-expansion-population}).
    In this case, significant adaptation-expansion association again arose with contemporary background ($p=0.03, \; \delta=0.25$);
    weaker associations, not significant under two-tailed alternative hypotheses, arose under prefatory and no biotic background ($\delta=0.20,\; 0.19$).
}
\label{fig:genome-expansion}

\end{figure}
